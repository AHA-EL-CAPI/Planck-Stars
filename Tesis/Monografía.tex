\documentclass[prb,aps,preprintnumbers,amsmath,amssymb]{article}

\usepackage{graphicx}
\usepackage{amsmath}
\numberwithin{equation}{section}
\usepackage{dcolumn}% Align table columns on decimal point
\usepackage{bm}% bold math
\usepackage[utf8]{inputenc}
\usepackage[spanish,es-tabla]{babel}

\usepackage{epstopdf}
%\usepackage[spanish]{babel}
%\usepackage[english]{babel}
%\usepackage[latin5]{inputenc}
%\usepackage{hyperref}
\usepackage[left=2cm,top=2.5cm,right=2cm,bottom =3.5cm,nohead,nofoot]{geometry}
\usepackage{braket}
\usepackage{datenumber}
%\newdate{date}{10}{05}{2013}
%\date{\displaydate{date}}

\begin{document}

\begin{center}
\Huge
Estudio de estrellas de Planck 

\vspace{3mm}
\Large Alejandro Hernández A.

\large
201219580


\vspace{2mm}
\Large
Director: Pedro Bargueño de Retes

\normalsize
\vspace{2mm}

\date{}
\end{center}


\normalsize
\section{Introducción}

\textbf{¡OJO! Hay que modificar esto apropiadamente, en principio, ahondar más en historia y en motivaciones para estudiar este tema}\\

\textbf{¡OJO! Hay que mencionar las métricas de Bardeen y de Vaidya}\\

El conocimiento actual del funcionamiento de la gravedad se basa en la Teoría General de la Relatividad de Einstein, cuya formulación matemática más general son las ecuaciones de campo, a saber:

\begin{equation}
\label{field equations}
R_{\mu \nu} - \frac{1}{2} R g_{\mu \nu} + \Lambda g_{\mu \nu} = 8 \pi T_{\mu \nu}
\end{equation}

donde $R_{\mu \nu}$ es el tensor de Ricci, $R$ es la curvatura escalar, $T_{\mu \nu}$ es el tensor energía-momento y $\Lambda$ es la constante cosmológica.\\

Muchas de las soluciones de \eqref{campo} no están bien definidas en todo el espacio-tiempo, tal y como ocurre, por ejemplo, con la métrica de Schwarzschild \eqref{sch}  en $r = 0$.

\begin{equation}
\label{sch}
ds^2 = -\left( 1 - \frac{2M}{r} \right) dt^2 + \left( 1 - \frac{2M}{r} \right)^{-1} dr^2 + r^2d\Omega ^2
\end{equation}

Una posible forma de solucionar estos problemas podría ser considerar correcciones cuánticas a la teoría gravitacional. Además, el hecho de que todos los campos no gravitacionales estén bien descritos por la teoría cuántica de campos, hace pensar que existe una teoría cuántica subyacente para la gravedad.\\ 

Una forma de implementar de forma efectiva correcciones cuánticas en la métrica de Schwarzschild es mediante la denominada métrica de Hayward modificada \cite{hayward,effective}, que viene dada por \eqref{reg-sch} 

\begin{equation}
\label{reg-sch}
ds^2 = -G(r)F(r) dt^2 + \frac{1}{F(r)} dr^2 + r^2d\Omega ^2
\end{equation}

donde

\begin{equation}
\label{reg-f}
F(r) = 1 - \frac{Mr^2}{r^3 + 2ML^2}
\end{equation}

\begin{equation}
\label{reg-g}
G(r) = 1 - \frac{\beta M \alpha}{\alpha r^3 + \beta M}
\end{equation}

con $\alpha$ y $\beta$ parámetros del sistema. Esta métrica (basada en los argumentos dados en \cite{hayward,bardeen}) representa lo que usualmente es referido en la literatura como \emph{Estrellas de Planck} \cite{planck stars} y  satisface las siguientes propiedades:

\begin{itemize}
\item $g_{00} = 1 - \frac{2M}{r}$ para $r \rightarrow \infty$, es decir, la m\'etrica es asint\'oticamente Schwarzschild.

\item Incluye correcciones efectivas de teoría cuántica de campos al potencial Newtoniano dadas por 

\begin{equation}
\label{new}
\Phi (r) = -\frac{M}{r} \left( 1 + \beta \frac{l_{p}^2}{r^2} \right) + \mathcal{O}(r^{-4}),
\end{equation}

donde $l_{p}$ en la longitud de Planck.
\item Permite una dilatación temporal finita entre $r = 0$ y $r = \infty$.

\item $g_{00} = 1 - \frac{r^2}{L^2} + o(r^3)$, es decir, es de Sitter para $r \rightarrow 0$.

\end{itemize}


Las motivaciones físicas para proponer la métrica \eqref{reg-sch}, además de las consecuencias y propiedades de la misma son de gran interés teórico y ejemplifican una forma particular de incluir efectos cuánticos de manera efectiva para regularizar la solución de Schwarzschild. \\

\textbf{¡OJO! Hacer conexión lógica con esta definición}\\

Se dice que un agujero es regular cuando los invariantes $R$, $R^{\mu,\nu}R_{\mu,\nu}$, $R^{\mu,\nu,\rho,\sigma}R_{\mu,\nu,\rho,\sigma}$ son finitos en todos los puntos del espacio-tiempo en cuestion.
 
\section{Preliminares}

\subsection{Horizontes de eventos y horizontes de Killing}

Existen diversos tipos de horizontes en relatividad general y las definiciones dadas en esta sección solo se remiten a aquellos horizontes que serán utilizados en el estudio posterior de las diversas métricas que se presentarán posteriormente.\\

Un horizonte de eventos es una frontera en el espacio-tiempo más allá de la cual los eventos que ocurran en su interior no pueden afectar a un observador externo. Para el caso particular de los agujeros negros, un horizonte de eventos puede ser entendido como la forntera a partir de la cual la velocidad necesaria para escapar del campo gravitacional del agujaro supera la velocidad de la luz.\\

Dada una métrica en coordenadas esféricas $(t,r,\theta,\phi)$, los horizontes de eventos de dicha métrica se localizan en los puntos donde $g^{rr}$ diverge.\\

Por otra parte, antes de definir un horizonte de Killing es preciso definir vectores de killing. Un campo vectorial de killing $X$ es, valga la redundancia, un campo vectorial que satisface

\begin{equation}
\nabla_\mu X_\nu + \nabla_\nu X_\mu = 0,
\end{equation}

es decir, que la derivada covvariante de este campo vectorial es antisimétrica (¡OJO!). Teniendo en cuenta lo anterior, un horizonte de killing se define como una hipersuperficie nula donde la norma de un vector de killing se hace cero.

\subsection{Condiciones de energía}

De vital importancia para el estudio de las métricas a lo largo de todo este documento son las denominadas condiciones de energía. Si bien en la mayoría de los casos resulta útil estudiar las ecuaciones de campo sin especificar la fuente de materia $T_{\mu \nu}$ en el espacio-tiempo descrito, en algunas ocasiones resulta interesante estudiar las propiedades de dichas ecuaciones que son válidas para diversas fuentes de materia. En esta última situación es fundamental imponer condiciones de energía que limiten la arbitrariedad de $T_{\mu \nu}$ con el fin de que sean fuentes razonables de energía y momento \cite{carroll book}.\\

Las formulaciones matemáticas de las múltiples  condiciones de energía se establecen a continuación:

\begin{itemize}
\item \textbf{Condición de energía débil (WEC):} Para todo vector timelike $t^\mu$ se satisface $T_{\mu \nu}t^{\mu}t^{\nu} \geq 0$. Para el caso particular de un fluido perfecto esta condición se traduce en $\rho \geq 0$ y $\rho + p \geq 0$.

\item \textbf{Condición de energía débil (NEC):} Como caso expecial de WEC, se exige que para cualquier vector nulo $l^\mu$ se tenga $T_{\mu \nu}l^{\mu}l^{\nu} \geq 0$. En el caso de un fluido perfecto, esta condición exige $\rho + p \geq 0$.

\item \textbf{Condición de energía dominante (DEC):} Esta condición incluye a WEC y se puede dividir en dos partes: la primera es exactamente igual a lo requerido por WEC; y la segunda exige que $T_{\mu \nu}T^{\nu}_{\lambda}t^{\mu}t^{\lambda} \geq 0$ para todo $t^{\mu}$ timelike. Para fluidos perfectos esta condición se traduce en $\rho \geq |p|$.

\item \textbf{Condición de energía nula dominante (NDEC):} Este requerimiento es igual a DEC salvo que se exige para vectores nulos en vez de para vectores timelike. Para fluidos perfectos, densidades negativas son permitidas siempre y cuando se satisfaga $p = -\rho$.

\item \textbf{Condición de energía fuerte (SEC):} Esta última condición exige que $T_{\mu \nu}t^{\mu}t^{\nu} \geq \frac{1}{2}T^{\lambda}_{\lambda}t^{\sigma}t_{\sigma}$ para todo vector $t^\mu$ timelike. Equivalentemente, se demanda que $\rho + p \geq 0$ y que $\rho + 3p \geq 0$ en el caso de un fluido perfecto. Cabe mencionar que la SEC no implica a WEC pero sí implica a NEC.
\end{itemize}

A lo largo de este documento, las condiciones estudiadas en las diversas métricas presentadas serán WEC, NEC, DEC y SEC.\\

¡OJO! Hawking y Ellis - Tipos de fluidos

\subsection{Métrica de Bardeen}

La métrica de Bardeen \cite{bardeen,borde-1994,borde-1996} fue una de las primeras en describir un agujero negro regular. En coordenadas esféricas $(t,r,\theta,\phi)$, la métrica de Bardeen se expresa como

\begin{equation}
\label{bardeen metric}
ds^2 = -\left( 1 - \frac{2mr^2}{(r^2 + q^2)^{3/2}} \right)dt^2 + \left( 1 - \frac{2mr^2}{(r^2 + q^2)^{3/2}} \right)^{-1}dr^2 + r^2d\Omega^2.
\end{equation}\\

Dicha métrica está inspirada en la de Reissner-Nordtröm por el hecho de que la componente temporal satisface

\begin{equation}
g_{tt} \sim_{r \to \infty} 1 - \frac{2m}{r} + \frac{3me^2}{r^3} + \mathcal{O}\left( \frac{1}{r^5} \right).
\end{equation}

Los dos primeros términos de la expansión corresponden, como es de esperarse, al comportamiento asintóticmente plano de la métrica de Bardeen, mientras que el último término mostrado permite identificar a $q$ como una especie de carga eléctrica, justo como en el caso de Reissner-Nordström (¡OJO! Def. de gtt en términos del potnecial - Límite newtoniano).\\

Más allá de lo mencionado anteriormente, y debido a que el elemento de línea \eqref{bardeen metric} es diagonal, es posbile obtener los horizontes de eventos de esta métrica al solucionar

\begin{equation}
g_{tt} = 1 - \frac{2mr^2}{(r^2 + q^2)^{3/2}} = 0.
\end{equation}

Y analizar las condiciones para la existencia de soluciones reales se concluye que la métrica \eqref{bardeen metric} posee horizontes de eventos cuando $q^2 < (16/27)m^2$. Este aspecto de la métrica de Bardeen también hace eco de una de las características fundamentales de la métrica de Reissner-Nordström en tanto que las propiedades globales del espacio-tiempo descrito tanto por la métrica de Reissner-Nordström como por la métrica de Bardeen dependen fuertemente de las magnitudes relativas entre $q$ y $m$.\\

Otro aspecto importante a tener en cuenta con respecto a la métrica de Bardeen son las condiciones de energía. Con la ayuda de \textit{Mathematica}, los componentes no nulos del tensor energía-momento (EMT) para la métrica \eqref{bardeen metric} son

\begin{equation}
\begin{split}
T_{tt} &= -\frac{6 m q^2 \left(2 m r^2-\left(q^2+r^2\right)^{3/2}\right)}{8 \pi \left(q^2+r^2\right)^4}\\
T_{rr} &= -\frac{6 m q^2}{8 \pi \left(q^2+r^2\right) \left(\left(q^2+r^2\right)^{3/2}-2 m r^2\right)}\\
T_{\theta \theta} &= \frac{m \left(9 q^2 r^4-6 q^4 r^2\right)}{8 \pi \left(q^2+r^2\right)^{7/2}}\\
T_{\phi \phi} &= \frac{3 m q^2 r^2 \sin ^2(\theta ) \left(3 r^2-2 q^2\right)}{8 \pi \left(q^2+r^2\right)^{7/2}}
\end{split}
\end{equation}

En particular, el EMT correspondiente a la métrica de Bardeen es diagonal y de a cuerdo a \cite{hawking}, el tensor anteriormente descrito corresponde a un fluido tipo I, la densidad de este fluido corresponde a $\rho = T_{tt}$ y las correspondientes presiones son $p_{1} = T_{rr}$, $p_{2} = T_{\theta \theta}$ y $p_{3} = T_{\phi \phi}$. Por ende, las condiciones de energía para la métrica de bardeen imponen las siguientes condiciones:

\begin{itemize}
\item La WEC requiere que $T_{tt} \geq 0$ y que $T_{tt} + T_{ii} \geq 0$ para $i = r,\theta,\phi$. Esto se satisface si
\begin{equation}
r>0\land \left((q<0\land m=0)\lor \left(q=0\land \left(m<\frac{\sqrt{r^2}}{2}\lor m>\frac{\sqrt{r^2}}{2}\right)\right)\lor (q>0\land m=0)\right).
\end{equation}
\item La NEC requiere que $T_{tt} + T_{ii} \geq 0$ para $i = r,\theta,\phi$.
\item La DEC requiere que $T_{tt} \geq |T_{ii}|$ para $i = r,\theta,\phi$.
\item La SEC requiere que $T_{tt} + T_{ii} \geq 0$ y $T_{tt} + 3T_{ii} \geq 0$ para $i = r,\theta,\phi$.
\end{itemize}


¡OJO! Explicar brevemente lo del cambio de topología para evitar la singularidad.\\

\textbf{¡OJO! La forma en la que se evita la singularidad en este y otros modelos que son considerados como Bardeen-like involucra una explicación que no he logrado entnder. Discutir con Pedro.}\\

¡OJO! Borde 1994 habla de global Cauchy surface condition. Tengo que leer un poco más para entenderlo.


\subsection{Métrica de Vaidya}

La métrica de Vaidya \cite{padmanabhan} es una generalización de la métrica de Schwarzschild que puede ser interpretada como un espacio-tiempo con una radiación saliente y esféricamente simétrica de partículas sin masa.\\

Para hallar esta métrica a partir de la de Schwarzschild es necesario pasar a coordenadas de Eddington-Finkelstein (¡OJO! Sería bueno ponerlo en un apéndice, junto con los ćalculos mencionados en otros OJO, poner lo de tortoise en el Gravitation), en las que la coordenada temporal $t$ es reemplazada por una nueva coordenada $u$ que satisface

\begin{equation}
dt = du + \frac{dr}{(1 - 2M/r)}
\end{equation}

\textbf{¡OJO! Hablar de la opción con el menos}\\

\textbf{¡OJO! Significado físico de u}\\

Con dicha transformación, el elemento de línea se convierte en 

\begin{equation}
\label{schw-tortoise}
ds^2 = - \left(1- \frac{2M}{r}\right)du^2 - 2dudr + r^2d\Omega^2
\end{equation}

Dado que nos interesa llegar a una métrica debida a fotones dirigidos radialmente hacia afuera (¡OJO! Entender esto), el elemento de línea anterior puede ser generalizado al considerar $M = M(u)$. Teniendo en cuenta lo anterior, y después de algunos cálculos (¡OJO! Cálculos en anexo, Padmanabhan), tenemos que el tensor energía-momento que genera la métrica \eqref{schw-tortoise} tiene como única componente distinta de cero $T_{uu} = -(M'/4 \pi r^2)$. Esto es exactamente igual al tensor que se obtendría para un rayo de fotones moviéndose radialmente hacia afuera con cuadrimomento $k_{a} = \nabla_{a}u$. El tensor energía-momento en este caso es $T^{ab} = -(M'/4 \pi r^2)k^{a}k^{b}$ (¡OJO!). Dado lo anterior, podemos interpretar la métrica de Vaidya como debida a una fuente esférica que pierde masa al emitir fotones de una forma simétrica.\\

Con respecto a las condiciones de energía, es claro que por la forma del EMT para la métrica de Vaidya, la única condición que vale la pena estudiar es WEC, la cual requiere $T_{uu} \geq 0$ y en términos de r requiere que $r > 0\ ,\ m'(u)\leq 0 $. Esta última condición coincide precisamente con la interpretación previamente mencionada sobre que la métrica deVaidya describe una estrella pierde masa conforme pasa el tiempo mediante la emisión de radiación en forma de fotones.\\

¡OJO! Hasta aquí Padmanabhan\\

¡OJO! Hablar más al tener en cuenta a Joshi.\\

\textbf{¡OJO! ¿Qué más incluir dentro de los prelimirares? ¿Dejar esta sección o ponerla en anexos?}\\

\section{Métrica de Hayward}

Teniendo todas las herramientas estudiadas en la sección anterior, es posible comenzar a estudiar el tema de interés para este trabajo.\\

¡OJO! Elaborar un poco más.\\

La métrica de Hayward está dada por 

\begin{equation}
ds^2 = \left( 1 - \frac{Mr^2}{r^3 + 2ML^2} \right) dt^2 + \left( 1 - \frac{Mr^2}{r^3 + 2ML^2} \right)^{-1} dr^2 + r^2d\Omega ^2
\end{equation}

El EMT asociado a esta métrica es

\begin{equation}
\begin{split}
T_{tt} &= \frac{3 m^2}{8 \pi} \left(\frac{4 \left(m^2 \left(r^2-2\right)+m \left(r^3-r^5\right)+r^6\right)}{\left(2 m+r^3\right)^4}-\left(\frac{m r^2}{2 m+r^3}-1\right) \left(\frac{2 \left(m-r^3\right)}{\left(2 m+r^3\right)^3}+\frac{9}{\left(18 m+r^3\right)^2}+\frac{16}{\left(32 m+r^3\right)^2}-\frac{8 \left(4 m-r^3\right) \left(r^3-m \left(r^2-2\right)\right)}{\left(2 m+r^3\right) \left(8 m+r^3\right)^2 \left(m \left(r^2-8\right)-r^3\right)}\right)\right)\\
T_{rr} &= \frac{3 m^2}{8 \pi} \left(\frac{16 \left(r^3-4 m\right)}{\left(8 m+r^3\right)^2 \left(m \left(r^2-8\right)-r^3\right)}-\frac{\frac{2 \left(m-r^3\right)}{\left(2 m+r^3\right)^3}+\frac{9}{\left(18 m+r^3\right)^2}+\frac{16}{\left(32 m+r^3\right)^2}-\frac{8 \left(4 m-r^3\right) \left(r^3-m \left(r^2-2\right)\right)}{\left(2 m+r^3\right) \left(8 m+r^3\right)^2 \left(m \left(r^2-8\right)-r^3\right)}}{1-\frac{m r^2}{2 m+r^3}}\right)\\
T_{\theta \theta} &= \frac{3 m^2 r^2}{8 \pi} \left(-\frac{2 \left(m-r^3\right)}{\left(2 m+r^3\right)^3}+\frac{9}{\left(18 m+r^3\right)^2}-\frac{16}{\left(32 m+r^3\right)^2}+\frac{8 \left(4 m-r^3\right) \left(r^3-m \left(r^2-2\right)\right)}{\left(2 m+r^3\right) \left(8 m+r^3\right)^2 \left(m \left(r^2-8\right)-r^3\right)}\right)\\
T_{\phi \phi} &= \frac{3 m^2 r^2 \sin ^2(\theta )}{8 \pi} \left(-\frac{2 \left(m-r^3\right)}{\left(2 m+r^3\right)^3}-\frac{9}{\left(18 m+r^3\right)^2}+\frac{16}{\left(32 m+r^3\right)^2}+\frac{8 \left(4 m-r^3\right) \left(r^3-m \left(r^2-2\right)\right)}{\left(2 m+r^3\right) \left(8 m+r^3\right)^2 \left(m \left(r^2-8\right)-r^3\right)}\right)
\end{split}
\end{equation}

Es claro que este EMT corresponde, de manera semejante que en la métrica de Bardeen, a un fluido tipo I y por tanto las condiciones de energía pueden ser inicialmente formuladas en los mismos términos.\\

¡OJO! Formular en términos de r.\\

¡OJO! Hacer conexión lógica con lo que sigue\\

La corrección efectiva de teoría cuántica de campos al potencial Newtoniano es

\begin{equation}
\label{new}
\Phi (r) = -\frac{M}{r} \left( 1 + \beta \frac{l_{p}^2}{r^2} \right) + \mathcal{O}(r^{-4}),
\end{equation}

donde $l_{p}$ en la longitud de Planck.\\

¡OJO! Elaborar un poco más\\

Para remover la singularidad en el origen, la métrica de Hayward satisface 

\begin{equation}
g_{tt}(r) \sim_{r \to 0} 1 - \frac{r^2}{l^2}
\end{equation}

Y teniendo en cuenta lo anterior, es posible definir una constante cosmológica para la métrica de Hayward, a saber

\begin{equation}
\Lambda = \frac{3}{l^2}
\end{equation}

\section{Métrica de Hayward modificada}

La métrica de Hayward modificada está dada por 
\begin{equation}
\label{reg-sch}
ds^2 = -G(r)F(r) dt^2 + \frac{1}{F(r)} dr^2 + r^2d\Omega ^2
\end{equation}

donde

\begin{equation}
\label{mod-hay-f}
F(r) = 1 - \frac{Mr^2}{r^3 + 2ML^2}
\end{equation}

\begin{equation}
\label{mod-hay-g}
G(r) = 1 - \frac{\beta M \alpha}{\alpha r^3 + \beta M}
\end{equation}

\section{Anexos}

¡OJO! Sería bueno hacer algo sobre las correcciones cuánticas\\

¡OJO! Cálculos Padmanabhan - Mathematica\\

¡OJO! Coordenadas Eddington-Finkelstein\\


\begin{thebibliography}{10}

\bibitem{1} ¡OJO! Usarlas todas en el texto

\bibitem{2} ¡OJO! Orden en el que aparecen en el texto.

\bibitem{hayward} Hayward, S.A.: Formation and Evaporation of regular black holes. Phys. Rev. Lett. \textbf{96}, 31103 (2006).

\bibitem{effective} De Lorenzo, T., Pacilio, C., Rovelli, C., Speziale, S.: On the effective metric of a Planck star. Gen. Relativ. Gravit. \textbf{47}, 41 (2015).

\bibitem{bardeen} Bardeen, J.M.: Non-singular general-relativistic gravitational collapse. In: Procceedings of International Conference GR5, Tbilisi, p. 174 (1968).

\bibitem{planck stars} Rovelli, C., Vidotto, F.: Planck Stars. Int. J. Mod. Phys. D. \textbf{23}, 1142026, (2014).


\bibitem{padmanabhan} Padmanabhan, T.: \textit{Gravitation: Foundations and Frontiers}. Cambridge University Press, Cambridge (2010).

\bibitem{inverno} d'Inverno, R. A.: \textit{Introducing Einstein's Relativity}. Oxford University Press, New York (1992).

\bibitem{hawking} Hawking, S.W., Ellis, G.F.R.: \textit{The Large Scale Structure of Space-time}, vol. 1, 20th edn. Cambridge University Press, Cambridge (1973).

\bibitem{gravitation} Misner, Charles W., Thorne, Kip S., Wheeler, John Archibald: \textit{Gravitation}, 1st ed. W. H. Freeman, San Francisco (1973).

\bibitem{borde-1994} Borde, A.: Open and closed universes, initial singularities, and inflation. Phy. Rev. D. \textbf{50}, 3692 (1994).

\bibitem{borde-1996} Borde, A.: Regular black holes and topology change. Phy. Rev. D. \textbf{55}, 7615 (1996).

\bibitem{carroll book} Carroll, Sean M.: \textit{Spacetime and Geometry: An introduction to General Relativity}, 1st ed. Addison Wesley, , San Francisco (2004). 

\end{thebibliography}


\end{document} 