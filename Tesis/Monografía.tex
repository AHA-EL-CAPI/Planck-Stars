\documentclass[16pt,a4paper]{article}
%\documentclass{article}

\usepackage{amssymb,amsmath}
\numberwithin{equation}{section}
\usepackage{dcolumn}% Align table columns on decimal point
\usepackage{bm}% bold math
\usepackage[utf8]{inputenc}
\usepackage[spanish,es-tabla]{babel}
\usepackage{hyperref}
\hypersetup{
    colorlinks=true,
    linkcolor=blue,
    urlcolor=red,
    linktoc=all
}
\usepackage[left=3cm,top=3cm,right=3cm,bottom =3cm,nohead,nofoot]{geometry}

\usepackage{enumerate}
\usepackage{graphicx}

\usepackage{amsthm}
\usepackage{mathtools}
\usepackage{setspace}

\newtheorem{theorem}{Teorema}[section]

\theoremstyle{definition}
\newtheorem{definition}{Definición}[section]


\begin{document}


\begin{titlepage}
	\centering
	\vspace*{\baselineskip} 
	\includegraphics[width=0.5\textwidth]{logo_uniandes}\par\vspace{1cm}
%	{\scshape\LARGE Universidad de Los Andes \par}
%	\vspace{1cm}
	{\scshape\Large Proyecto de Grado en Física\par}
	\vspace{1.5cm}
	\rule{\textwidth}{1.6pt}\vspace*{-\baselineskip}\vspace*{2pt}
	\rule{\textwidth}{0.4pt}\\[\baselineskip]
	{\huge\bfseries Estudio de estrellas de Planck\par}
	\rule{\textwidth}{0.4pt}\vspace*{-\baselineskip}\vspace{3.2pt}
	\rule{\textwidth}{1.6pt}\\[\baselineskip]
	\vspace{2cm}
	{\Large\itshape Alejandro Hernández A.\par}
	\vfill
	Director:\par
	Dr. Pedro Bargueño de Retes 
	\vfill

% Bottom of the page
%	{\large 9 de Marzo de 2016\par}
\end{titlepage}

\doublespacing
\newpage
\vspace*{\fill}
\noindent
\huge{Abstract:}
\newline\newline
\normalsize
In the last decades there has been efforts to include quantum theory in general relativity in order to remove the black hole singularities. Loop quantum gravity provides an alternative to do this by considering the possibility that the energy of collapsing star can condensate into a highly compressed core with a density of the order of the Planck scale. In this case, the gravitational collapse do not lead to a singularity but to a new phase in the life of a star called a Planck star, in which the gravitational attraction is balanced by a strong quantum pressure. In this work, the metric describing the Planck stars is studied in detail, emphasizing the multiple calculations behind such a metric. To do so, the relevant energy conditions for regular space-times are established, important metrics such as the Bardeen metric and the Vaidya metric are studied, and finally the so called Hayward metric is studied as the metric that describes the aforementioned Planck stars. In the end of this work, a physical interpretation for this metric is given and some predictions of this model are analyzed.
\vspace*{2cm}
\newline\newline
\huge{Resumen:}
\newline\newline
\normalsize
En las últimas décadas se ha intentado incluir la teoría cuántica en relatividad general con el fin de remover las singularidades de los agujeros negros. Una alternativa para esto, proveniente de loop quantum gravity, considera la posibilidad de que la energía de una estrella que colapsa puede condensarse en un núcleo altamente comprimido con una densidad del orden de la escala de Planck. En este caso, el collapso gravitacional no conlleva a una singularidad sino a una nueva fase en la vida de la estrella, denominada estrella de Planck, en la que la atracción gravitacional es balanceada por una fuerte presión cuántica. En este trabajo se estudia detalladamente la métrica que describe las estrellas de Planck, haciendo particular énfasis en los múltiples cálculos innatamente asociados a dicha métrica. Para ello, se establecen las condiciones de energía de interés para métricas regulares, se estudian métricas previas importantes, tales como la métrica de Bardeen y la de Vaidya, para posteriormente estudiar la métrica de Hayward, la cual describe las susodichas estrellas de Planck. Finalmente, se realiza una interpretación física de la métrica y se analizan algunas predicciones de este modelo.
\vspace*{\fill}

\newpage
\large
\tableofcontents
\newpage

\normalsize
\section{Introducción}

Desde la publicación del famoso artículo \textit{The Foundation of the General Theory of Relativity} %\cite{einstein} 
por Albert Einstein en el año de 1916, la búsqueda de soluciones exactas a sus ecuaciones de campo ha sido uno de los principales motores de investigación en relatividad general (GR). La primera solución matemática exacta de dichas ecuaciones fue la métrica de Schwarzschild, que describe la geometría generada en el exterior de un cuerpo esféricamente simétrico, sin carga y sin momento angular. Además de la anterior, otra métrica que presenta simetría esférica es la de Reissner Nordström (métrica RN), la cual describe un cuerpo cargado sin rotación. Estas métricas comparten una característica particular, a saber, ambas poseen una singularidad física en $r = 0$. Esto dificulta el tratamiento de ese punto tanto en el espaciotiempo de Schwarzschild como en el de Reissner Norström, puesto que la teoría no es aplicable en el susodicho punto singular.\\

Teniendo en cuenta lo anterior, los puntos singulares de un espaciotiempo dado dejan entrever fallas de la teoría general de la relatividad puesto que no permite describir las propiedades físicas de los mismos. Con el fin de evadir estas dificultades, se dice que un agujero es regular, o sin puntos singulares, cuando los invariantes de curvatura $R$, $R^{\mu \nu}R_{\mu \nu}$, $R^{\mu \nu \rho \sigma}R_{\mu \nu \rho \sigma}$ son finitos en todos los puntos del espaciotiempo en cuestion. Dado esto, desde hace más de cincuenta años el estudio de espaciotiempos (agujeros negros) regulares ha sido de gran interés para múltiples físicos y matemáticos, entre ellos, James M. Bardeen. \\

En el año de 1968, Bardeen fue el primero en proponer una solución regular de las ecuaciones de campo de Einstein. Si bien el documento \cite{bardeen} en el que se propuso el modelo no está disponible, una revisión detallada de la métrica de Bardeen puede ser consultada en \cite{borde1994}. La importancia de esta métrica no solo se debe a que fue la primera de su estilo sino también a que fue crucial para orientar las futuras investigaciones sobre agujeros negros regulares. Al respecto, Ayón-Beato y García propusieron una interpretación física del modelo de Bardeen en términos de electrodinámica no lineal  \cite{ayon-beato2000} y construyeron muchos otras métricas correspondientes a agujeros negros regulares \cite{ayon-beato2005,ayon-beato1999-1,ayon-beato1999-2,ayon-beato1999-3}.\\

Ayende de lo anterior, trabajos más recientes \cite{hayward2006,lorenzo,rovelli} tratan de generalizar el modelo que propuso Bardeen de regularización de agujeros negros, de tal forma que se incluyan correcciones efectivas de teoría cuántica de campos en relatividad general. En estos modelos se definen el concepto de Estrella de Planck, que corresponde a un métrica regular, esféricamente simétrica, asintóticamente plana, que satisface la condición de energía débil y que no posee horizontes de Cauchy.\\

Precisamente, el objetivo de este trabajo es entender el proceso mediante el cual se formulan las métricas correspondientes a estrellas de Planck, entender el mecanismo mediante el cual dichas métrica evaden la singularidad en $r = 0$, además de tratar de interpretar físicamente el tensor de energía-momento (SET) asociado a las mismas. Teniendo en cuenta lo previamente mencionado, la organización de este documento se da a continuación. En la Sec. \ref{preliminaries section} se establecen las condiciones de energía de forma usual en términos de SET y se establece la metodología general que se empleará para estudiar todas las métricas presentadas. En la Sec. \ref{previous metrics section} se estudian en detalle las métricas de Bardeen y de Vaidya, las cuales constituyen la base sobre la cual se definen las estrellas de Planck. Posteriormente, en la Sec. \ref{planck stars section} se estudia a fondo la métrica que define las aludidas estrellas de Planck y se da la interpretación física de las mismas. Finalmente, en la sección \ref{conclusions} se resumen los resultados del trabajo y se habla sobre las correcciones cuánticas que introducen las estrellas de Planck en relatividad general.\\


\newpage
\section{\label{preliminaries section} Preliminares}

%¡OJO! Ejemplos cuando se hable de cada métrica.\\
%¡OJO! Blau Notes \\
%
%Existen diversos tipos de horizontes en relatividad general y las definiciones dadas en esta sección solo se remiten a aquellos horizontes que serán utilizados en el estudio posterior de las diversas métricas que se presentarán posteriormente.
%
%\subsection{Horizontes de eventos}
%
%Un horizonte de eventos es una frontera en el espacio-tiempo más allá de la cual los eventos que ocurran en su interior no pueden afectar a un observador externo. Para el caso particular de los agujeros negros, un horizonte de eventos puede ser entendido como la forntera a partir de la cual la velocidad necesaria para escapar del campo gravitacional del agujaro supera la velocidad de la luz.\\
%
%Dada una métrica en coordenadas esféricas $(t,r,\theta,\phi)$, los horizontes de eventos de dicha métrica se localizan en los puntos donde $g^{rr}$ diverge.
%
%\subsection{Horizontes de Killing}
%
%Antes de definir un horizonte de Killing es preciso definir vectores de killing. Un campo vectorial de killing $X$ es, valga la redundancia, un campo vectorial que satisface
%
%\begin{equation}
%\nabla_\mu X_\nu + \nabla_\nu X_\mu = 0,
%\end{equation}
%
%es decir, que la derivada covvariante de este campo vectorial es antisimétrica (¡OJO!). Teniendo en cuenta lo anterior, un horizonte de killing se define como una hipersuperficie nula donde la norma de un vector de killing se hace cero.
%
%\subsection{Horizontes aparentes}
%
%Los horizontes aparentes son \cite{blau}.

	\subsection{Condiciones de energía}

De vital importancia para el estudio de las métricas a lo largo de todo este documento son las denominadas condiciones de energía. Si bien en la mayoría de los casos resulta útil estudiar las ecuaciones de campo sin especificar la fuente de materia $T_{\mu \nu}$ en el espacio-tiempo descrito, en algunas ocasiones resulta interesante estudiar las propiedades de dichas ecuaciones que son válidas para diversas fuentes de materia. En esta última situación es fundamental imponer condiciones de energía que limiten la arbitrariedad de $T_{\mu \nu}$ con el fin de que sean fuentes razonables de energía y momento \cite{carroll}.\\

Las formulaciones matemáticas de las múltiples  condiciones de energía se establecen a continuación:

\begin{itemize}
\item \textbf{Condición de energía débil (WEC):} Para todo vector timelike $t^\mu$ se satisface $T_{\mu \nu}t^{\mu}t^{\nu} \geq 0$. Para el caso particular de un fluido perfecto esta condición se traduce en $\rho \geq 0$ y $\rho + p \geq 0$.

\item \textbf{Condición de energía débil (NEC):} Como caso expecial de WEC, se exige que para cualquier vector nulo $l^\mu$ se tenga $T_{\mu \nu}l^{\mu}l^{\nu} \geq 0$. En el caso de un fluido perfecto, esta condición exige $\rho + p \geq 0$.

\item \textbf{Condición de energía dominante (DEC):} Esta condición incluye a WEC y se puede dividir en dos partes: la primera es exactamente igual a lo requerido por WEC; y la segunda exige que $T_{\mu \nu}T^{\nu}_{\ \lambda}t^{\mu}t^{\lambda} \geq 0$ para todo $t^{\mu}$ timelike. Para fluidos perfectos esta condición se traduce en $\rho \geq |p|$.

\item \textbf{Condición de energía fuerte (SEC):} Esta última condición exige que $T_{\mu \nu}t^{\mu}t^{\nu} \geq \frac{1}{2}T^{\lambda}_{\ \lambda}t^{\sigma}t_{\sigma}$ para todo vector $t^\mu$ timelike. Equivalentemente, se demanda que $\rho + p \geq 0$ y que $\rho + 3p \geq 0$ en el caso de un fluido perfecto. Cabe mencionar que la SEC no implica a WEC pero sí implica a NEC.
\end{itemize}

Como se verá más adelante, las únicas dos condiciones de energía que realmente interesan para el estudio de agujeros negros regulares son la WEC y la SEC, esto por el significado físico que hay destrás de estas condiciones, a saber, la WEC establece la no-negatividad de la densidad de energía para cualquier observador, mientras que la SEC alude al carácter atractivo de la fuerza gravitatoria.\\

\subsection{Procedimiento General}

Teniendo en cuenta el Teorema de Birkhoff \cite{gravitation} y la derivación hecha en \cite[Cap. 7]{carroll-lecture-notes} de la métrica de Schwarzschild, la forma más general de una métrica estática y esféricamente simétrica es

\begin{equation}
\label{general static spherical}
ds^2 = -f(r)dt^2 + \frac{1}{f(r)}dr^2 + r^2d\Omega^2,
\end{equation}

donde la función $f(r)$ puede ser escrita como

\begin{equation}
\label{general f}
f(r) = 1 - \frac{2m(r)}{r}.
\end{equation}

Ahora bien, dado que nos interesa estudiar agujeros negros regulares, es preciso aclarar que la SEC se debe violar en algún lugar dentro del horizonte \cite{zaslavskii}, a la vez que la WEC o la DEC pueden ser satisfechas en todo el espaciotiempo generado \cite{dymnikova2004}. Al considerar el elemento de línea \eqref{general static spherical} junto con \eqref{general f}, es posbile escribir los componente de SET como \cite{vanegas-weak} signatura y métrica ortogonal

\begin{equation}
\label{wec set comp}
\begin{aligned}
T^{t}_{\ t} = T^{r}_{\ r} = \frac{2}{8 \pi r^2} \frac{dm(r)}{dr},\\ T{^\theta}_{\theta} = T^{\phi}_{\ \phi} = \frac{1}{8 \pi r} \frac{d^2m(r)}{dr^2},
\end{aligned}
\end{equation}

y por ende, se concluye que la WEC puede expresarse de manera equivalente en términos de la función de masa $m(r)$ mediante las siguientes desigualdades

\begin{equation}
\label{mass wec ineq}
\begin{aligned}
\frac{1}{r^2}\frac{dm(r)}{dr} &\geq 0,\\
\frac{2}{r}\frac{dm(r)}{dr} &\geq \frac{d^2m(r)}{dr^2}.
\end{aligned}
\end{equation}

Como se mencionó previamente, la DEC incluye por continuidad a la WEC. Esto permite argumentar que, por simplicidad, la única condición de energía que se va a exigir para los agujeros negros regulares presentados en secciones poseriores será la WEC.

\newpage
\section{\label{previous metrics section} Métricas relevantes}

A continuación se realiza un estudio detallado de las métricas de Bardeen y de Vaidya, las cuales son cruciales para el estudio posterior de la métrica que describe el espaciotiempo generado por las estrellas de Planck.

\subsection{\label{bardeen section} Métrica de Bardeen}

Como se mencionó en la introducción, la métrica de Bardeen \cite{bardeen} fue la primera en describir un espaciotiempo regular. En coordenadas esféricas $(t,r,\theta,\phi)$, la métrica de Bardeen se expresa como

\begin{equation}
\label{bardeen metric}
ds^2 = -\left( 1 - \frac{2mr^2}{(r^2 + g^2)^{3/2}} \right)dt^2 + \left( 1 - \frac{2mr^2}{(r^2 + g^2)^{3/2}} \right)^{-1}dr^2 + r^2d\Omega^2,
\end{equation}

con $m$ y $q$ parámetros no negativos, en algunos casos llamados parámetros de regularización. Claramente, para esta métrica tenemos que 

\begin{equation}
m(r) = \frac{2mr^3}{(r^2 + g^2)^{3/2}},
\end{equation}

por tanto, la WEC se satisface automáticamente en cualquier punto del espacio tiempo puesto que 

\begin{align}
\begin{aligned}
\frac{1}{r^2}\frac{dm(r)}{dr} &= \frac{6 m g^2}{\left(g^2+r^2\right)^{5/2}},\\
\frac{2}{r}\frac{dm(r)}{dr} &= \frac{12 m g^2 r}{\left(g^2+r^2\right)^{5/2}},\\
\frac{d^2m(r)}{dr^2} &= \frac{6 m \left(2 g^4 r-3 g^2 r^3\right)}{\left(g^2+r^2\right)^{7/2}}.
\end{aligned}
\end{align}

El carácter regular de la métrica de Bardeen es evidente a partir del hecho de que, por un lado

\begin{equation}
f_{bardeen}(r) \underset{r \to 0}{\sim} 1 - \frac{2mr^2}{g^3} + \mathcal{O}(r^4),
\end{equation}

es decir, posee un comportamiento de de Sitter en la vecindad de $r = 0$, y por otro, los escalares de curvatura

\begin{equation}
\label{bardeen scalars}
\begin{gathered}
R = \frac{6 g^2 m \left(4 g^2-r^2\right)}{\left(g^2+r^2\right)^{7/2}},\\
R_{\mu \nu}R^{\mu \nu} = \frac{18 g^4 m^2 \left(8 g^4-4 g^2 r^2+13 r^4\right)}{\left(g^2+r^2\right)^7},\\
R_{\mu \nu \sigma \rho}R^{\mu \nu \sigma \rho} = \frac{12 m^2 \left(8 g^8-4 g^6 r^2+47 g^4 r^4-12 g^2 r^6+4 r^8\right)}{\left(g^2+r^2\right)^7},
\end{gathered}
\end{equation}

son finitos en todo punto  del espaciotiempo.\\

Ahora bien, debido a que el elemento de línea \eqref{bardeen metric} es diagonal, es posbile obtener los horizontes de eventos de esta métrica al solucionar

\begin{equation}
f_{bardeen}(r) = 1 - \frac{2mr^2}{(r^2 + g^2)^{3/2}} = 0.
\end{equation}

Y analizar las condiciones para la existencia de soluciones reales se concluye que la métrica \eqref{bardeen metric} describe un agujero negro (es decir, posee horizontes de eventos) para $g^2 \leq (16/27)m^2$: cuando $g^2 < (16/27)m^2$ existen dos horizontes de eventos $r_{+},r_{-}$ y cuando $g^2 = (16/27)m^2$ solo hay un horizonte de eventos degenerado $r_{+} = r_{-}$. En este último caso se dice que el agujero negro es extremal, tal y como en la métrica RN, en tanto que posee un único horizonte de eventos. Es importante recalcar que en cualquier caso, los horizontes de eventos para el agujero negro de Bardeen tan solo constituyen singularidades debidas al sistema de coordenadas puesto que ya se mencionó que los invariantes de curvatura (\ref{bardeen scalar}-\ref{bardeen riemann scalar}) son finitos en todo el espaciotiempo. La extensión de la métrica de Bardeen más allá de los horizontes de eventos se puede hacer al cambiar a las coordenadas de Eddington-Finkelstein salientes, en términos de las cuales la métrica es bien comportada incluso en el caso del agujero negro extremal. \\

Considere nuevamente la componente temporal $g_{tt}$ de la métrica \eqref{bardeen metric}. Note que

\begin{equation}
f_{bardeen}(r) \underset{r \to \infty}{\sim} 1 - \frac{2m}{r} + \frac{3mg^2}{r^3} + \mathcal{O}\left( \frac{1}{r^5} \right).
\end{equation}

Los dos primeros términos de la expansión corresponden, como es de esperarse, al comportamiento asintóticamente plano de la métrica de Bardeen, y por tanto, es posible identificar al parámetro $m$ como la masa de la configuración descrita por esta métrica. No obstante, el término $1/r^3$ impide asociar el parámetro  de regularización $g$ con algún tipo de carga de Coulomb, a diferencia de la métrica de Reisner Norström, en la que el parámetro $q$ es automáticamente asociado a una carga eléctrica al comparar el SET de esta métrica con el del campo eléctrico para una carga puntual. Este hecho fue el principal obstáculo para la interpretación física del parámetro $g$ en la métrica de Bardeen hasta que Ayón-Beato y García propusieron una interpretación en términos de electrodinámica no-lineal \cite{ayon-beato2000}.\\

El modelo de Ayón-Beato y García parte la suposición de que la dinámica de la métrica de Bardeen está regida por

\begin{equation}
\mathcal{S} = \int dv \left( \frac{1}{16 \pi}R - \frac{1}{4 \pi}\mathcal{L}(F) \right),
\end{equation}

donde $R$ es el escalar de curvatura; $\mathcal{L}$ es una función de $F = \frac{1}{4}F_{\mu \nu}F^{\mu \nu}$, con $F_{\mu \nu} = 2\nabla_{[\mu}A_{\nu]}$ el tensor electromagnético; y $v$ es la coordenada avanzada de Eddington-Finkelstein. Las ecuaciones de Eisntein acopladas con electrodinámica no-lineal resultantes de la acción $\mathcal{S}$ son\\

\begin{equation}
\label{euler lagrange eqns}
\begin{aligned}
G_{\mu}^{\ \nu} = 2(\mathcal{L}_{F} F_{\mu \lambda} F^{\nu \lambda} - \delta_{\mu}^{\ \nu} \mathcal{L}),\\
\nabla_{\mu}(\mathcal{L}_{F} F^{\alpha \mu}) = 0,
\end{aligned}
\end{equation}

donde $\mathcal{L}_{F} \equiv \partial \mathcal{L}/\partial F$. La fuente electrodinámica no-lineal usada por los autores para derivar la métrica de Bardeen es 

\begin{equation}
\label{nonlinear bardeen}
\mathcal{L}(F) = \frac{3}{2sg^2}\left( \frac{\sqrt{2g^2F}}{1 + \sqrt{2g^2F}} \right)^{5/2},
\end{equation}

donde $s \equiv |g|/2m$; m la masa de la configuración descrita; y $g$ un parámetro de integración. Además, los autores consideran el ansatz para el campo electromagnético 

\begin{equation}
F_{\mu \nu} = 2 \delta^{\theta}_{\ \lbrack\mu} \delta^{\varphi}_{\ \nu\rbrack} B(r, \theta),
\end{equation}

y al tener en cuenta todas las consideraciones previamente mencionadas, Ayón-Beato y García concluyen que el parámetro $g$ corresponde a la carga de un monopolo magnético, esto es 

\begin{equation}
\frac{1}{4 \pi} \int_{S^{\infty}}\mathbf{F} = \frac{g}{4 \pi} \int_{0}^{\pi}\int_{0}^{2\pi}\sin (\theta) d\theta d\varphi = g.
\end{equation}

Todo lo anterior permite entender la métrica \eqref{bardeen metric} como un acople de electrodinámica no-lineal a la relatividad general, que físicamente corresponde a un monopolo magnético autogravitante de carga $g$. Esta interpretación de la métrica de Bardeen hace eco de una de las características fundamentales de la métrica RN en tanto que las propiedades globales del espacio-tiempo descrito tanto por la métrica de Reissner-Nordström como por la métrica de Bardeen dependen fuertemente de las magnitudes relativas entre el parámetro de carga y el parámetro de masa.\\

Ahora bien, hasta el momento solo se ha lidiado con la interpretación física de la métrica de Bardeen \eqref{bardeen metric} y se ha establecido el carácter regular de la misma en términos de la finitud de los invariantes de curvatura. No obstante, es preciso ahondar un poco más en el último aspecto mencionado, dado que es el principal motivo para estudiar dicha métrica en este trabajo.\\

En el año 1965, Roger Penrose publicó un famoso artículo \cite{penrose} acerca de singularidades en el espaciotiempo. Una revisión más reciente del mismo puede ser consultada en \cite{senovilla2015}. La formulación precisa del teorema de singularidad de Penrose, junto con algunas definiciones preliminares al mismo, se establecen a continuación. 

\theoremstyle{definition}
\begin{definition}
\label{trapped surfaces}
	Sea $\mathcal{M}$ un espaciotiempo. Sea $\mathcal{S} \subset \mathcal{M}$ una hipersuperficie de $\mathcal{M}$.
	\begin{enumerate}[i]
		\item $\mathcal{S}$ se llama hipersuperficie de Cauchy si es intersecada exactamente una vez por cualquier curva inextendible tipo tiempo. En otras palabras, $\mathcal{S}$ representa todo el espacio en un instante de tiempo. 
		
		\item $\mathcal{S}$ se llama futuramente atrapada si cualquier geodésica nula de tipo futuro que emana ortogonalmente de la misma tiene divergencia negativa ($\theta < 0$) en la superficie.
		
		\item $\mathcal{S}$ se llama eventualmente futuramente atrapada si la divergencia de cualquier geodésica nula tipo futuro se vuelve negativa en algún punto del futuro de la superficie.
	\end{enumerate}
\end{definition}
\
\begin{theorem}[Penrose, 1965]
\label{penrose sing thm}
Si un espaciotiempo satisface que

\begin{enumerate}[i]
\item contiene una hipersuperficie de Cauchy $\Sigma$ no-compacta y conexa,

\item contiene una superficie cerrada futuramente atrapada (future-trapped surface),

\item cumple con la condición de convergencia nula $R_{\mu \nu}u^{\mu}u^{\nu} \geq 0$ para cualquier vector nulo $u^{\mu}$ (o equivalentemente, cumple la NEC),
\end{enumerate}

entonces dicho espacio tiempo posee geodésicas incompletas de tipo nulo futuro.
\end{theorem}
\

En otras palabras, el teorema anterior establece que un espaciotiempo dado tiene una singularidad si las tres condiciones enunciadas previamente se satisfacen. En relación con lo anterior, y basado en las consideraciones hechas en \cite{borde1994,borde1996}, la métrica de Bardeen es frecuentemente aludida como contraejemplo a la posibilidad de poder probar la existencia de singularidades en el interior de agujeros negros sin asumir la contenencia de una hipersuperficie de Cauchy o la condición de energía fuerte (condiciones I y III del teorema).\\

En tanto que la métrica de Bardeen está inspirada en la métrica RN, la pregunta evidente en este caso es: ¿cómo logra la métrica de Bardeen evitar la singularidad física en $r = 0$ de la métrica RN? La respuesta a esta pregunta fue establecida desde una perspectiva netamente matemática por Arvind Borde en \cite{borde1994,borde1996}. Con el fin de entender los planteamientos de Borde, es preciso considerar la extensión maximal de la métrica de Bardeen. En la fig. \ref{fig: bardeen diagram} está representada una parte de la extensión maximal del mencionado espaciotiempo.\\

\begin{figure}[h!]
	\centering
	\includegraphics[width=0.35\textwidth]{bardeenDiagram}
	\caption{Estructura global de una porción del agujero negro de Bardeen. Imagen tomada de \cite{borde1994}.}
	\label{fig: bardeen diagram}
\end{figure}

Como se ha mencionado previamente, las propiedades globales, y en particular, la estructura causal de la métrica de Bardeen es similar a la de la métrica de Reissner Nordström, de ahí la similitud entre los diagramas de Penrose-Carter correspondientes a ambas métricas. Cada punto en el interior del espaciotiempo representado en la fig. \ref{fig: bardeen diagram} es una 2-esfera, las líneas sólidas y los círculos respresentan regiones en el infinito, y las líneas punteadas representan el origen ($r = 0$) del sistema de coordenadas esféricas (Las dos líneas eituqetadas con $r = 0$ representan dos "patches" diferentes de coordenadas). La región sombreada representa parte del interior del agujero negro, de tal forma que señales enviadas desde esa región no pueden escapar al infinito futuro tipo nulo $i^{+}$. Dado a que esa región, en particular, está dentro del horizonte de enventos del agujero, hay superficies atrapadas $\mathcal{T}$ en la misma. De acuerdo a Borde, el caráter regular de la métrica de Bardeen se debe a que, en la región señalada, es posible que los rayos de luz se ``envuelvan'' alrededor del universo universo, en otras palabras, aunque los sistemas de geodésicas nulas salientes o entrantes de $\mathcal{T}$ convergen, estas geodésicas convergen a puntos $r = 0$ de ``patches'' distintos. De hecho, la superficie atrapada $\mathcal{T}$ y su cono de luz futuro $E^{+}(\mathcal{T})$ yacen en el desarrolo de Cauchy futuro de la superficie $\mathcal{S}_{2}$ y la singularidad en $r = 0$ se evita única y exclusivamente debido a que $\mathcal{S}_{2}$ es compacta.\\

En otras palabras, las secciones espaciales de un agujero negro de Bardeen evolucionan desde una región en la que no son compactas (por ejemplo, la superficie $\mathcal{S}_{1}$ de la fig. \ref{fig: bardeen diagram}), hacia una región donde son compactas, y por ende, el unvierso cambia su topología de abierto a cerrado. Estos razonamientos son formalizados por Bardeen mediante el siguiente teorema, en el cual se establece que dicho cambio de topología es necesario para la existencia de agujeros negros regulares que obedecen la condición de energía débil.\\

\begin{theorem}[Borde, 1996]
	\label{borde reg thm}
	Suponga que una espaciotiempo $\mathcal{M}$ satisface que
	
	\begin{enumerate}[i]
		\item contiene una superficie eventualmente futuramente atrapada $\mathcal{T}$,
		
		\item obedece la condición de convergencia nula,
		
		\item el cojunto de geodésicas nulas futuras es completo,
		
		\item su futuro causal es simple, con $E^{+}(X) \neq \emptyset,\ \forall X \subset \mathcal{M}$.
	\end{enumerate}
	
	Entonces hay una sección espacial compacta en el futuro de $\mathcal{T}$.
\end{theorem}

Una prueba (a simple vista, sencilla) de este teorema puede ser consultada en \cite{borde1996}. Es importante mencionar que, si bien no hay ninguna alución a un cambio de topología en el teorema, es usual que un espaciotiempo de agujero negro contenga una regiión en infinito y, por ende, es de esperar que ``comience'' con una sección espacial no compacta $\mathcal{S}$ y por lo menos una superficie eventualmente futuramente atrapada en el futuro de la misma. En estos términos, el teorema de Borde muestra que, bajo condiciones muy generales, para que un agujero negro pueda ser regular se requiere que el espaciotiempo del cual hace parte desarrolle una sección espacial compacta en el futuro de $\mathcal{S}$, es decir, que la topología debe cambiar de abierta a cerrada. Una consecuencia inmediata de este cambio es que dicho espacio no puede ser globalmente hiperbólico.\\

Para tratar de entender un poco más el teorema en sí, es preciso explicar con más detalle cada una de las hipótesis del mismo.\\

La primera hipótesis es de esperar que sea satisfecha en el interior de cualquier agujero negro. Como ya se mencionó, una superficie futuramente atrapada es una en la cual todas geodésicas nulas futuramente dirigidas que emanan ortogonalmente de dicha superficie tienen divergencia negativa en la misma. La diferencia fundamental con las superficies enventualmente futuramente atrapadas es que solo se requiere que la divergencia se vuelva negativa en el futuro de la superficie a lo largo de cualquier geodésica. La intención de este debilitamiento de la noción de una superficie atrapada, como Borde mismo lo afirma, se hace con el fin de incluir situaciones en las que el agujero negro esté creciendo dado a materia que cae en el mismo. Dentro de un agujero negro es de esperar que hayan2-superficies cerradas tipo espacio que son eventualmente  futuramente atrapadas. Esto puesto que de no ser así, habría congruencias de geodésicas con divergencia no negativa a lo largo de la susodicha superficie, las cuales perfectamente podrían escapar hacia el futuro. Esto es lo que ocurre por ejemplo para $e^2 = m^2$ en la métrica RN.\\

La segunda hiótesis se sigue automáticamente de la condición de energía débil, la cual es la única que exigimos que cumplan los modelos discutidos en este trabajo. Es importante notar que sin algún tipo de condición de convergencia, se pueden tener agujeros negros regulares sin ningún cambio de topología.\\

La tercera hipótesis es satisfecha automáticamente en cualquier métrica que describa un espacio-tiempo-regular, que son precisamente las métricas que estamos estudianto en este documento.\\

Finalmente, para explicar la última hipótesis se requiere de algunas definciones relacionadas con la estructura causal del espacio tiempo, las cuales están inspiradas en el espacio-tiempo de Schwarszchild. Sean $x$, $y$ dos puntos en una variedad $M$.
\begin{itemize}
\item $x$ precede cronolóogicamente a $y$, denotado $x \ll y$, si existe una curva tipo tiempo dirigida hacia el futuro desde $x$ hasta $y$.

\item $x$ estrictamente precede causalmtne a $y$, denotado $x \leq y$, si existe una curva tipo tiempo o tipo nula dirigida hacia el futuro desde $x$ hasta $y$.

\item $x$ precede causalmente a $y$, denotado $x \prec y$, si $x$ estrictamente precede causalmente a $y$ o si $x = y$.
\end{itemize}

Teniendo en cuenta lo anterior, para $x \in M$ se define
\begin{itemize}
\item El futuro cronológico de $x$, denotado $I^+(x)$, como el conjunto de todos los puntos $y \in M$ tales que $x$ precede cronológicamente a $y$, esto es
\begin{equation*}
I^+(x) \coloneqq \left\lbrace y \in M | x \ll y \right\rbrace
\end{equation*}
\item El pasado cronológico de $x$, denotado $I^-(x)$, como el conjunto de todos los puntos $y \in M$ tales que $y$ precede cronológicamente a $x$, a saber
\begin{equation*}
I^-(x) \coloneqq \left\lbrace y \in M | y \ll x \right\rbrace
\end{equation*}
\end{itemize}

Análogamente se define
\begin{itemize}
\item El futuro causal de $x$, denotado $J^+(x)$, como el conjunto de todos los puntos $y \in M$ tales que $x$ precede causalmente a $y$. De forma compacta
\begin{equation*}
J^+(x) \coloneqq \left\lbrace y \in M | x \prec y \right\rbrace
\end{equation*}
\item El pasado causal de $x$, denotado $J^+(x)$, como el conjunto de todos los puntos $y \in M$ tales que $y$ precede causalmente a $x$. Con la simbología anterior, esto significa
\begin{equation*}
J^-(x) \coloneqq \left\lbrace y \in M | y \prec x \right\rbrace
\end{equation*}
\end{itemize}

También es preciso recordar la definición del cono de luz. Considerando el evento $p$ como un pulso de luz en un tiempo $t_0$, todos los eventos que pueden ser alcanzados por este pulso desde $p$ conforman el cono de luz futuro de $p$, mientras que los eventos que pueden enviar un pulso de luz a $p$ forman el cono de luz pasado de $p$. El cono de luz de algún evento $p$ divide todos los eventos del espacio-tiempo en 5 categorías, las cuales son
\begin{itemize}
\item Los eventos en el cono de luz futuro de $p$.

\item Los eventos en el cono de luz pasado de $p$.

\item Los eventos dentro del cono de luz futuro de $p$, que corresponden a los eventos que pueden ser afectados por partículas emitidas desde $p$.

\item Los eventos dentro del cono de luz pasado de $p$, los cuales pueden emitir partículas y afectar a $p$.

\item Los eventos que no pueden afectar causalemente a $p$.
\end{itemize}

Con lo anterior, si definimos a $E^+(x)$ como el cono de luz futuro de un punto $x \in M$, es posible demostrar que $E^+(x) \subset \dot{I}^+(x)$, donde $\dot{I}^+(x)$ denota la frontera del futuro cronológico de $x$. No obstante, en general estos dos conjuntos no son iguales. Espacio-tiempos para los cuales $E^+(x) = \dot{I}^+(x)$ se llaman causalemente simples y físicamente esto significa básicamente que su estructura causal es semejante a la del espacio-tiempo de Schwarszchild.\\

Basado en las definiciones anteriores, el espacio-tiempo generado por la métrica de Bardeen (y por métricas inspiradas en la de Bardeen) satisfacen la condición de ser causalmente simples. Además, es necesario hacer alguna suposición sonre causalidad para que se tenga el argumento de cambio de topología, puesto que de no hacerlo, existen ejemplos basados en el universo de Gödel, en los que no se tiene ni singularidad ni cambio de topología.\\

Habiendo explicado cada una de las hipótesis del teorema de Borde es preciso mencionar que no se tiene una comprensión completa del mismo, es decir, que no se logró comprender exactamente la necesidad de cada una de sus hipótesis para establecer ese resultado, ni su conexión directa entre el cambio de topología que plantea este teorema y el carácter regular de un espacio-tiempo dado. No obstante, que este teorema básicamente provee un converso para el teorema de Penrose, se tratará de desmenuzar el significado específico del teorema de Borde en un trabajo futuro.\\

Para finalizar esta sección sobre la métrica de Bardeen es preciso mencionar que, si bien lo que hemos hecho hasta el momento es el primer paso para estudiar las estrellas de Planck, dichas estrellas son descritas por espaciotiempos que no son estáticos en absoluto. Por ende, es necesario desarrollar herramientas que permitan estudiar la evolución dinámica inminente de las susodichas estrellas. La forma en que se abordará espaciotiempos que varían con el tiempo será a través de la famosa métrica de Vaidya.

\subsection{Métrica de Vaidya}

La métrica de Vaidya \cite{padmanabhan} es una generalización de la métrica de Schwarzschild que puede ser interpretada como un espaciotiempo con una radiación saliente y esféricamente simétrica de partículas sin masa.\\

Esta métrica puede ser hallada a partir de la de Schwarzschild 

\begin{equation}
ds^2 = -\left( 1 - \frac{2m}{r} \right) + \left( 1 - \frac{2m}{r} \right)^{-1}dr^2 + r^2d\Omega^2,
\end{equation}
y para ello es necesario pasar a coordenadas de Eddington-Finkelstein, en las que la coordenada temporal $t$ es reemplazada por una nueva coordenada $u$ que satisface

\begin{equation}
dt = du + \frac{dr}{(1 - 2m/r)}.
\end{equation}

Para la interpretación física de esta nueva coordenada, note que $u = cte$ corresponde a curvas que satisfacen la ecuación $dr/dt = +\left( 1 -2m/r \right)$. Las líneas radiales nulas en el espaciotiempo de Schwarzschild corresponden a curvas con $d\theta = d\phi = ds = 0$ y en términos de $r$ y $t$ satisfacen $(dr/dt)^2 = \left( 1 -2m/r \right)^2$. La raíz positiva de esta ecuación representa líneas de tipo nulo dirigidas radialmente hacia afuera (la raíz negativa, a rayos dirigidos radialmente hacia adentro).\\

Con dicha transformación, el elemento de línea se convierte en 

\begin{equation}
\label{schw-tortoise}
ds^2 = - \left(1- \frac{2m}{r}\right)du^2 - 2dudr + r^2d\Omega^2.
\end{equation}

Dado que nos interesa llegar a una métrica debida a fotones dirigidos radialmente hacia afuera, podemos considerar una generalización de \eqref{schw-tortoise} a una métrica dinámica en la que m varía respecto a la nueva coordenada temporal $u$, esto es, $m = m(u)$. Por ende Teniendo en cuenta lo anterior, el tensor energía-momento que genera la métrica \eqref{schw-tortoise} tiene como única componente distinta de cero $T_{uu} = -(m'/4 \pi r^2)$, con $m' = dm(u)/du$. Esto es exactamente igual al tensor que se obtendría para un rayo de fotones moviéndose radialmente hacia afuera con cuadrimomento $k_{a} = \nabla_{a}u$. El EMT en este caso es $T^{ab} = -(m'/4 \pi r^2)k^{a}k^{b}$. Dado lo anterior, podemos interpretar la métrica de Vaidya como debida a una fuente esféricamente simétrica que pierde masa al emitir fotones en la dirección radial.\\

Con respecto a las condiciones de energía, es claro que por la forma del EMT para la métrica de Vaidya, la única condición que vale la pena estudiar es WEC, la cual requiere $T_{uu} \geq 0$ y en términos de r requiere que $m'(u)\leq 0 $. Esta última condición coincide precisamente con la interpretación previamente mencionada sobre que la métrica de Vaidya describe una estrella pierde masa conforme pasa el tiempo mediante la emisión de radiación en forma de fotones.\\

Ahora bien, a parte de la presencia de radiación, la métrica de Vaidya posee una característica fundamental que la diferencia radicalmente de la métrica de Schwarzschild, y esto es el tipo de superficie (horizonte) en $r = 2m(u)$. En el espacioteimpo de Schwarzschild, $r = 2m$ representa una superficie de tipo nulo que denota el conocido horizonte de eventos. Por su parte, a continuación se verá que en el espaciotiempo de Vaidya la superficie $r = 2m(u)$ es de tipo espacio y por ende no puede ser asociada a un horizonte de eventos. De hecho, esta superficie conforma lo que usualmente es denominado en la literatura \cite{blau,griffiths2009,wald2010} como horizonte aparente.\\


Un horizonte aparente se define técnicamente como una hipersuperficie que separa las regiones que poseen superficies atrapadas de las regiones que no contienen este tipo de superficies. No obstante, en aras de dar una definción matemáticamente más concreta de horizontes aparentes es necesario formalizar matmáticamente la definición de hipersuperficies nulas y definir el concepto de divergencia de campos vectoriales.\\

Como es sabido, una hipersuperficie es una subvariedad $\Sigma$ n-dimensional de una variedad $M$ n+1-dimensional \cite{blau}. Una descripción particularmente útil para estudiar la geometría de estas subvariedades se da a través de los embebimientos $\Phi: \Sigma \to M$. Si $\Sigma$ es equipado con coordenadas $y^\alpha$ y $M$ con coordenadas $x^\alpha$, un embebimiento $\Phi$ está dado explícitamente al especificar el punto en $M$ con coordenadas $x^\alpha$ que corresponde a un punto en $\Sigma$ con coordenadas $y^\alpha$, en otras palabras, un embebimiento está dado por las ecuaciones paramétricas 
\begin{equation*}
x^\alpha = x^\alpha(y^\alpha).
\end{equation*}

Hay dos condiciones que deben ser satisfechas para que un mapa $\Phi$ sea un embebimiento, a saber
\begin{itemize}
	\item $\Phi$ debe ser inyectivo.
	\item El jacobiano de $\Phi$, es decir, la matriz $(n+1)\times n$
	\begin{equation*}
	E^{\alpha}_{a} = \frac{\partial x^\alpha}{\partial y^a},
	\end{equation*}
	
	tiene rango maximal n.
\end{itemize}

Dado que los vectores $E^{\alpha}_{a}$ son linealmente independientes y tangentes a la imagen de $\Sigma$ en $M$, los vectores normales a $\Sigma$ (es decir, orthonormales a $\Sigma$) están caracterizados por 
\begin{equation*}
g_{\alpha \beta}E^{\alpha}_{a}\xi^{\alpha} = E^{\alpha}_{a}\xi_{\beta} = 0,
\end{equation*}

con $g_{\alpha \beta}$ la métrica de $M$. Es importante notar que si $\xi^{\alpha}$ es un vector normal, entonces $f\xi^{\alpha}$ también es normal a $\Sigma$ para cualquier función escalar $f$ que es distinta de cero en $\Sigma$.\\

Teniendo en cuenta lo anterior, los vectores normales a $\Sigma$ permiten establecer la siguiente caracterización de hipersuperficies
\begin{align*}
\Sigma\ es\ tipo\  
\begin{cases}
\ espacio\ & si\ \xi^{\alpha}\xi_{\alpha}<0,\\
\ tiempo\ & si\ \xi^{\alpha}\xi_{\alpha}>0,\\
\ nulo\ & si\ \xi^{\alpha}\xi_{\alpha}=0.\\
\end{cases}
\end{align*}

Cuando $\Sigma$ no es nula, la libertad en la elección de $f$ puede ser usada para normalizar el vector normal a un vector normal de longitud unitaria $\pm 1$. Esta codición de normalización determina el vector normal unitario $N^{\alpha}$ de forma única salvo una elección de signo, esto es
\begin{align*}
N^{\alpha} = \frac{\xi^{\alpha}}{|\xi^{\alpha}\xi_{\alpha}|^{1/2}} \Rightarrow N^{\alpha}N_{\alpha} =\ 
\begin{cases}
\ -1 & \ si\ \Sigma\ es\ tipo\ espacio,\\
\ +1 & \ si\ \Sigma\ es\ tipo\ tiempo.\\
\end{cases}
\end{align*}

Evidentemente, esta normalización es imposible para las hipersuperficies tipo nulo. No obstante, una elección natural y conveniente de un vector normal a la hipersuperficie es 
\begin{equation*}
l_\alpha = -\partial_\alpha S,
\end{equation*}

donde el signo ha sido escogido de tal forma que $l^\alpha$ está orientado hacia el futuro para una fución real-valuada $S$ que aumenta hacia el futuro y en términos de la cual la hipersuperficie $\Sigma$ se define como
\begin{equation*}
\Sigma = \lbrace x \in M:\ S(x) = 0 \rbrace.
\end{equation*}

De esta forma, los demás vectores normales a $\Sigma$ son de la forma
\begin{equation*}
\xi^\alpha = fl^\alpha,
\end{equation*}

para alguna función $f$ que no se anula sobre $\Sigma$.\\

Ahora bien, con las herramientas anteriores es posible definir la métrica inducida $h_{ab}$ en una hipersuperficie (nula), y la utilidad de $h_{ab}$ recae en que a partir de la misma, la identificación del tipo al cual pertenece una hipersuperfcie dada es automático, permitiéndonos diferenciar fácilmente entre horizontes ade eventos y horizontes aparentes.\\

Dado que en general las geodésicas nulas que son los generadores de una hipersuperficie nula están naturalmente asociados a dicha hipersuperficie, es conveniente adaptar las coordenadas $y^\alpha$ en $\Sigma$ a $l^\alpha$ al escoger las coordenadas de tal forma que 
\begin{equation*}
y^\alpha = (v = \lambda, y^k),
\end{equation*}

donde $\lambda$ es el parámetro (no necesariamente afín) a lo largo de las geodésicas y $y^k$ coresponde a las coordenadas espaciales que denotan las geodésicas nulas individuales. En estas coordenadas, los vectores tangentes $E_\alpha$ a la hipersuperficie nula están dados por
\begin{equation*}
E^{\alpha}_{v} = \frac{\partial x^\alpha}{\partial \lambda} = l^\alpha \hspace{.5cm}, \hspace{.5cm} E^{\alpha}_{k} =\frac{\partial x^\alpha}{\partial y^k},
\end{equation*}

y por ende la métrica inducida
\begin{equation*}
h_{ab} = g_{\alpha \beta} E^{\alpha}_{a} E^{\alpha}_{b},
\end{equation*}

tiene los componentes 

\begin{equation*}
h_{vv} = g_{\alpha \beta} l^\alpha l^\beta = 0 \hspace{.5cm},\hspace{.5cm} h_{vk} = g_{\alpha \beta} l^\alpha E^{\beta}_{k} = 0, \hspace{.5cm},\hspace{.5cm} h_{km} \equiv s_{km} = g_{\alpha \beta} E^{\alpha}_{k} E^{\beta}_{m},
\end{equation*}

donde $h_{vk} = 0$ se sigue puesto que por construcción los vectores $E^{\alpha}_{k}$ son tangentes a la superficie mientras que por definición $l^\alpha$ es normal a la hipersuperficie y en particular normal a los vectores tangentes a la misma. \\

Dado lo anterior, la métrica es degenerada (lo cual es característico para las hipersuperficies nulas) y el elemento de línea toma la forma
\begin{equation*}
ds^2|_{\substack{\Sigma}} = s_{km}dy^k dy^m = g_{\alpha \beta} E^{\alpha}_{k} E^{\beta}_{m} dy^k dy^m.
\end{equation*}

Es importante notar que esta forma de la métrica es independiente de si el parámetro $\lambda$ es o no el parámetro afín de las geodésicas nulas, puesto que la escogencia de este parámetro puede ser tenida en cuenta al cambiar $l^\alpha \to \xi^\alpha = fl^\alpha$ para alguna ellección apropiada de $f$, de tal forma que siempre se tiene $h_{vv} = h_{vk} = 0$.

Con la finalidad de acercarnos a la definción de divergencia de un campo vectorial, la última herramienta requerida para la diferenciación entre horizontes de eventos y horizontes aparentes es el vector auxiliar $n$ que satisface las siguientes propiedades
\begin{equation}
\label{prop div}
n_\alpha n^\alpha = l_\alpha l^\alpha = 0 \hspace{.5cm},\hspace{.5cm} n_\alpha E^{\alpha}_{k} =  n_\alpha E^{\alpha}_{k} = 0 \hspace{.5cm},\hspace{.5cm} n_\alpha l^\alpha = -1,
\end{equation}

que corresponde a un vector normal en $\Sigma$ pero no tangente, que es linealmente independiente tanto a $E^{\alpha}_{k}$ como a $l^\alpha$.\\

Como último preludio para definir la divergencia de campos vectoriales, es preciso mencionar la conocida ecuación de Raychaudhuri para congruencias geodésicas tipo tiempo, a saber,
\begin{equation*}
\frac{d}{d\tau}\theta = -\frac{1}{3}\theta^2 - \sigma^{\alpha \beta}\sigma_{\alpha \beta} + \omega^{\alpha \beta}\omega_{\alpha \beta}-R_{\alpha \beta}u^\alpha u^\beta,
\end{equation*}

donde $\tau$ es el parámetro de las geodésicas tipo tiempo $u^\alpha$, y $\omega_{\alpha \beta}$, $\sigma_{\alpha \beta}$ y $\theta$ son conocidos como el tensor de rotación, el tensor de cizalla y la expansión de la congruencia de las geodésicas $u^\alpha$.\\

Centrando la atención en la expansión $\theta$, esta cantidad puede ser expresada como
\begin{equation*}
\theta = \frac{1}{2}h^{\alpha \beta}L_ug_{\alpha \beta},
\end{equation*}

donde $L_u$ denota la derivada de Lie a lo largo del campo vectorial $u$. Y en términos del elemento de volumen $\sqrt{h}$ se tiene que \cite{blau}
\begin{equation*}
\theta = \frac{1}{\sqrt{h}}\frac{d}{d\tau}\sqrt{h},
\end{equation*}

de tal forma que $\theta$ mide el cambio de volumen transversal de la congruencia con elemento de volumen $\sqrt{h}$. lA importancia de lo mencionado previamente radica en que al estudiar la ecuación de Raychadhuri para congruencias de geodésicas nulas afínmente parametrizadas, la expanción $\theta$ se generaliza a un tensor $\theta_l$ denominado expansión o divergencia del campo vectorial $l$, relacionado con el campo vectorial tangencial $l$ a las geodésicas. Matméticamente se define la divergencia como 
\begin{equation*}
\theta_l = \nabla^\alpha l_\alpha,
\end{equation*}

y en términos de la derivada de Lie tenemos que
\begin{equation*}
\theta_l = \frac{1}{\sqrt{s}}L_l\sqrt{s},
\end{equation*}

lo cual permite la interpretación análoga de la divergencia $\theta_l$ como la medida del cambio del elemento de área transversal $\sqrt{s}$ de la congruencia. Por su puesto, la definición de divergencia no está limitada al campo vectorial $l^\alpha$ sino que también se define análogamente para el campo auxiliar $n^\alpha$, esto es
\begin{equation*}
\theta_l = \frac{1}{\sqrt{s}}L_n\sqrt{s}.
\end{equation*}

Al trabajar con simetría esférica, la expresión matemática de las divergencias se simplifican, y al tener en cuenta el elemento de área transversal $\sqrt{s} = r^2 \sin \theta$, se tiene que 
\begin{equation}
\label{divergences}
\theta_l = \frac{2}{r}l^r \hspace{.5cm},\hspace{.5cm} \theta_l = \frac{2}{r}n^r.
\end{equation}

La utilidad de toda la maquinaria desarrollada para la definición de las divergencias de campos vectoriales \eqref{divergences} es evidente a partir de la siguiente discusión.\\

En aras de estudiar la geometría del espacio-tiempo correspondiente a la métrica de Vaidya es preciso ubicar los horizontes entre la región donde los denominados rayos de luz salientes y orientados hacia el futuro son realmente salientes en el sentido en que se mueven a valores de $r$ más grandes, $dr/d\tau >0$; y la región donde los rayos "salientes'' en realidad se mueven hacia valores más pequeños de $r$, $dr/d\tau <0$. Esto se puede ver directamente desde la condición
\begin{equation*}
-f(v,r)dv + 2dr = 0 \hspace{.5cm} \Leftrightarrow \hspace{.5cm} 2\frac{dr}{dv} = f(v,r),
\end{equation*}

donde $f(v,r) = 1 - 2m(v)/r$. De esta manera, dependiendo del valor de $r$ se tiene que
\begin{align*}
2\frac{dr}{dv} = f(v,r)\ 
\begin{cases}
\ > 0\ para\ r > 2m(v): rayos\  salientes\\  
\ < 0\ para\ r > 2m(v): rayos\  entrantes.\\
\end{cases}
\end{align*}

La discusión anterior puede ser puesta en términos más geométricos al introducir los campos vectoriales radiales nulos para esa geometría, los cuales son
\begin{equation*}
n = -\partial_r \hspace{.5cm},\hspace{.5cm} l = \partial_v + \frac{1}{2}f(v,r)\partial_r,
\end{equation*}

los cuales satisfacen las condiciones \eqref{prop div}. Para determinar sus divergenias, se pueden considerar las 2-esferas $S = S_{v,r}$ de $r$ y $v$ constantes. La geometría intrínseca está caracterizada por la métrica inducida, en particular, por el elemento de volumen inducido
\begin{equation*}
\sqrt{s} = r^2 \sin \theta.
\end{equation*}

Dada la simetría esférica, la geometría extrínseca de las 2-esferas puede ser caracterizada en términos del cambio fraccional del elemento de área a lo largo de $l$ y $n$, es decir, por las expresiones en \eqref{divergences}.\\

Ahora bien, la no trivialidad de la superficie $r = 2m(v)$ para la métrica de Vaidya está codificada en $\theta_l$, y en este caso se tiene que
\begin{equation*}
\theta_l = \frac{r - 2m(v)}{r^2}.
\end{equation*} 

Por tanto
\begin{align*}
\theta_l\ 
\begin{cases}
\ >0\ para\ r>2m(v),\\
\ =0\ para\ r=2m(v),\\
\ <0\ para\ r<2m(v).
\end{cases}
\end{align*}

Dado que en general para una 2-superficie S con $\theta_n < 0$ tenemos que 
\begin{align*}
S\ es\ 
\begin{cases}
\ no-atrapada\ & para\ \theta_l>0,\\
\ marginalmente\ atrapada\ & para\ \theta_l=0,\\
\ atrapada\ & para\ \theta_l<0,
\end{cases}
\end{align*}

se concluye que para la métrica de Vaidya tenemos que 
\begin{align*}
S_{v,r}\ es\ 
\begin{cases}
\ no-atrapada\ & para\ r>2m(v),\\
\ marginalmente\ atrapada\ & para\ r=2m(v),\\
\ atrapada\ & para\ r<2m(v).
\end{cases}
\end{align*}

Considerando lo previamente mencionado, el horizonte de eventos del agujero negro de Schwarzschild puede ser caracterizado, con las nuevas herramientas, como la hipersuperficie nula que consiste de las esferas marginalmente atrapadas con $r = 2m$, esto es
\begin{equation*}
\mathcal{T} = \cup_{\substack{v}}S_{v,r = 2m}.
\end{equation*}

Una definición análoga puede ser hecha en el caso de Vaidya  
\begin{equation*}
\mathcal{T} = \cup_{\substack{v}}S_{v,r = 2m(v)}.
\end{equation*}

Sin embargo, la sutil diferencia en la dependencia del radio en la coordenada $v$ hace dudar sobre el carácter de dicha hipersuperficie. La pregunta automática en este caso es ¿qué tipo de hipersuficie representa $\mathcal{T}$ en el espacio-tiempo de Vaidya? Para responder a esta pregunta, es útil conciderar la métrica inducida en esta hipersuperficie, la cual, a partir de \eqref{schw-tortoise}, está dada por\\
\begin{equation}
\label{ind vaiya}
ds^2|_{\substack{f(v,r)=0}} = 4m'(v)dv^2 + (2m(v))^2d\Omega^2,
\end{equation}

por ende, si la condición de energía nula $m'(v) \geq 0$ se satisface (como lo hace en la métrica de Hayward estudiada posteriormente), $\mathcal{T}$ es tipo espacio a menos que $m'(v) = 0$, caso en el cual la superficie es tipo nulo. Esto pone en evidencia la radical diferencia que ya se había referido entre la métrica de Schwarzschild y la métrica de Vaidya, a saber, que la hipersuperficie $r = 2m(v)$ deja de ser un horizonte de eventos puesto que ya no correponde a una hipersuperficie tipo nulo, sino una tipo espacio.\\

Finalmente, este tipo de superficies constituyen los afamados horizontes aparentes, los cuales corresponden a un tubo marginalmente atrapado de acuerdo a \cite{blau}. Los tubos marginalmente atrapados son simplemente 3-superficies foliadas por superficies marginalmente atrapadas, esto es, superficies con $\theta_n <0$ y $\theta_l = 0$. He aquí la utlidad de la divergencia de campos vectoriales, particularmente, de los campos vectoriales tangentes y normales a la hipersuperficie de interés, pues en terminos de sus signos se caracteriza completamente el tipo de dicha superficie, y por ende, la existencia o no de horizontes de eventos para un espacio-tiempo dado.	

\newpage
\section{\label{planck stars section} Métrica de Hayward}

Teniendo todas las herramientas estudiadas en las seccioes anteriores, es posible comenzar a estudiar el tema de interés para este trabajo, a saber, la métrica correspondiente a las estrellas de Planck.

\subsection{Métrica de Hayward estática}

Si bien es posible encontrar métricas como la de Bardeen, es decir, esféricamente simétricas, estáticas, asintóticamente planas, con centros regulares, cuyos EMT es físicamente razonable, en particular que satisface la WEC. No obstante, dichos espaciotiempos típicamente soy considerados como no-físicos en tanto que poseen horizontes de Cauchy. No obstante, si consideramos la evaporación de uno de esos agujeros negros, el horizonte de cauchy pasa a user tan real como el horizonte de eventos.  Precisamente, las estrellas de Planck son descritas por métricas regulares dno estáticas (dinámicas) que satisfacen todas las propiedades anteriormente mencionadas, y que además poseen la característica adicional de incluir correcciones de teoría cuántica de campos en relatividad general.\\

Con el fin de entender paulatinamente el concepto de estrella de Planck, es preciso dividir su estudio entre la parte estática y la parte dinámica. La parte estática es descrita por la métrica de Hayward, la cual está dada por 

\begin{equation}
\label{hayward metric}
ds^2 = -\left( 1 - \frac{2mr^2}{r^3 + 2ml^2} \right) dt^2 + \left( 1 - \frac{2mr^2}{r^3 + 2ml^2} \right)^{-1} dr^2 + r^2d\Omega ^2,
\end{equation}

donde $m$ y $l$ son parámetros cuya interpretación física se explicará más adelante. Claramente, para esta métrica tenemos que

\begin{equation}
\label{hayward mass}
m(r) =  \frac{mr^3}{r^3 + 2ml^2}.
\end{equation}

Por ende, la métrica de Hayward satisface inmediatamente la WEC en tanto que 

\begin{equation}
\begin{aligned}
\frac{1}{r^2}\frac{dm(r)}{dr} &= \frac{12 l^2 m^2}{\left(2 l^2 m+r^3\right)^2},\\
\frac{2}{r}\frac{dm(r)}{dr} &= \frac{24 l^2 m^2 r}{\left(2 l^2 m+r^3\right)^2},\\
\frac{d^2m(r)}{dr^2} &= \frac{48 m^2 \left(l^4 m r-l^2 r^4\right)}{\left(2 l^2 m+r^3\right)^3}.
\end{aligned}
\end{equation}

Como en el caso de Bardeen, la métrica de Hayward describe un espaciotiempo regular puesto que, por un lado, 

\begin{equation}
f_{hayward}(r) \underset{r \to 0}{\sim} 1 - \frac{r^2}{l^2} + \mathcal{O}(r^5),
\end{equation}

es decir, es de Sitter en la vecindad de $r =0$, y por otro, sus invariantes de curvatura

\begin{equation}
\label{hayward scalars}
\begin{gathered}
R = \frac{24 l^2 m^2 \left(4 l^2 m-r^3\right)}{\left(2 l^2 m+r^3\right)^3},\\
R_{\mu \nu}R^{\mu \nu} = \frac{288 m^4 \left(8 l^8 m^2-4 l^6 m r^3+5 l^4 r^6\right)}{\left(2 l^2 m+r^3\right)^6},\\
R_{\mu \nu \sigma \rho}R^{\mu \nu \sigma \rho} = \frac{48 m^2 \left(32 l^8 m^4-16 l^6 m^3 r^3+72 l^4 m^2 r^6-8 l^2 m r^9+r^{12}\right)}{\left(2 l^2 m+r^3\right)^6},
\end{gathered}
\end{equation}

permanecen finitos en todo el espaciotiempo.\\

Ahora bien, al analizar los ceros de $f_{hayward}(r)$ revela una masa crítica de $m_{*} = (3\sqrt{3}/4)l$ y un radio crítico de $r = \sqrt{3}l$. Dado lo anterior, si $m < m_{*}$, la métrica \ref{hayward metric} describe un espaciotiempo regular con la misma estructura causal que cualquier espaciotiempo plano; si $m = m_{*}$ tenemos un agujero negro regular con un único horizonte de eventos degenerado; y si $m > m_{*}$ tenemos un agujero negro con dos horizontes de eventos. Esto se ve reflejado en la fig. \ref{fig: fhayward analysis}. 

\begin{figure}[h!]
	\centering
	\includegraphics[width=0.75\textwidth]{fhaywardAnalysis}
	\caption{Posibles casos de agujeros negros para la métrica de Hayward con para un valor fijo del parámetro $l$ y diferentes valores del parámetro $m$. La estructura causal de este espaciotiempo es similar a la de Reissner Norström.}
	\label{fig: fhayward analysis}
\end{figure}

Evidentemente, la pregunta natural con respecto a las anteriores consideraciónes es cuál es la fuente física que da origen a esta métrica. Para contestarla, note que 

\begin{equation}
f_{hayward}(r) \underset{r \to \infty}{\sim} 1 - \frac{2m}{r} + \frac{4l^2m^2}{r^4} + \mathcal{O}(r^{-5}),
\end{equation}

luego, al igual que en el caso de Bardeen, podemos interpretar el parámetro $m$ como la masa de la distribución que genera el elemento de línea \eqref{hayward metric}. No obstante, dada la presencia del término $r^{-4}$, es imposible interpretar el parámetro $l$ en términos de una carga coulombiana. De hecho, Hayward plantea que $l$ da aproximadamente la escala de longitud debajo de la cual los efectos de cuánticos en gravedad comienzan a ser dominantes, razón por la cual es de esperar que $l$ sea la longitud de Planck o del mismo orden de magnitud, aunque longitudes mayores no son excluídas. Cabe mencionar que sería interesante hallar una interpretación física para el parámetro $l$ tal como se mostró para el parámetro $g$ en el caso de Bardeen, es decir, hallando una fuente de electrodinámica no-lineal para la función de masa \eqref{hayward mass}, y por ende, para el elemento de línea \eqref{hayward metric}. Si hay suficiente tiempo, dicha interpretación se colocará en la sección de anexos.\\


\subsection{Métrica de Hayward dinámica}

Para la parte dinámica de las estrellas de Planck es preciso considerar la coordenada avanzada de Eddignton-Finkelstein entrantes, las cuales se obtienen a partir de

\begin{equation}
dt = dv - \frac{dr}{(1 - 2m(r)/r)}.
\end{equation}

En términos de la nueva coordenada, el elemento de línea \eqref{hayward metric} se transforma en 

\begin{equation}
\label{hayward-tortoise}
ds^2 = -\left( 1 - \frac{2mr^2}{r^3 + 2ml^2} \right) dv^2 + 2dvdr + r^2d\Omega ^2,
\end{equation}

y al generalizar la función de masa a $m = m(v)$ tenemos que el EMT asociado a \eqref{hayward-tortoise} es

\begin{equation}
\begin{aligned}
T_{t}^{\ t} &= T_{r}^{\ r} = -\frac{12 l^2 m^2}{\left(2 l^2 m+r^3\right)^2}, \\
T_{\theta}^{\ \theta} &= T_{\phi}^{\ \phi} = -\frac{24 l^2 m^2 \left(l^2 m-r^3\right)}{\left(2 l^2 m+r^3\right)^3},\\
T_{v}^{\ r} &= \frac{2r^4m'}{(r^3 + 2l^2m)^2},
\end{aligned}
\end{equation}

donde $m' = dm/dv$. Las componentes $T_{t}^{\ t}$, $T_{r}^{\ r}$, $T_{\theta}^{\ \theta}$, $T_{\phi}^{\ \phi}$ son iguales a los de la métrica de Hayward estática, la única novedad es la componente $T_{v}^{\ r}$, la cual describe puramente la radiación entrante que fue incorporada al modelo al considerar $m = m(v)$.\\

En analogía con la métrica de Vaidya, es posible que los horizontes de eventos de la métrica de Hayward estática se ``convierten'' en horizontes aparentes en el caso dinámico. 

¡OJO! 1/8$\pi$ en los EMT

\subsection{Métrica de Hayward modificada}

Con respecto a lo mencionado sobre la escala de Planck, estudios posteriores al artículo de Hayward (véase \cite{lorenzo}) han hecho explícito el hecho de que la presencia del paŕametro de longitud $l$ en la métrica de Hayward necesariamente refleja la inclusión de correcciones efectivas de teoría cuántica de campos en relatividad general. Al respecto, De Lorenzo \emph{et al.} plantean que, para que las métricas que describen espaciotiempos regulares a partir de modificaciones de la métrica de Schwarzschild sean físicamente plausibles, se requieren dos condiciones: que dichas métricas incorporen las correcciones efectivas de teoría cuántica  de campos al potencial Newtoniano, y una dilatación temporal no trivial entre un observador en $r \to \infty$ y un observador en $r = 0$. Precisamente, esta sección se encarga de estudiar con más detalle estos aspectos.\\

De acuerdo a De lorenzo \emph{et al.}, la mayoría de las métricas correspondientes a modelos de agujeros negros regulares poseen dos características que carecen de sentido físico, a saber, que un reloj en el centro regular no presenta ningún retraso con respecto a un reloj en el infinito, y que no reroducen las correcciones efectivas de toería cuańtica de campos al potencial Newtoniano \cite{lorenzo}. Por estos motivos, los autores proponen una modificación a la métrica de interés, en este caso la de Hayward, que corrige estos ``defectos'' de dicha métrica.\\

La corrección efectiva de teoría cuántica de campos al potencial Newtoniano es

\begin{equation}
\label{newF}
\Phi (r) = -\frac{M}{r} \left( 1 + \beta \frac{l_{p}^2}{r^2} \right) + \mathcal{O}(r^{-4}),
\end{equation}

donde $l_{p}$ en la longitud de Planck.\\

Antes de establecer la modificiación hecha por lor mencionados autores, es preciso deducir la expresión que relaciona el potencial Newtoniano con la métrica del espacio-tiempo en cuestión. La relación se realizará por medio del cálculo del límite newtoniano.\\

Queremos determinar las condiciones bajo las cuales las ecuaciones relativistas de movimiento, es decir, las ecuaciones de campo de Einstein, se reducen a la ecuación lineal de movimiento de la mecánica newtoniana
\begin{equation*}
\frac{d^2}{dt^2}\vec{x} = - \vec{\nabla}\phi,
\end{equation*}

con $\phi$ el potencial gravitacional
\begin{equation*}
\phi = - \frac{m}{r}.
\end{equation*}

Por ende, queremos determinar las circunstancias bajo las cuales son válidas las ecuaciones de Newton. Para ello, hay dos condiciones que son evidentemente necesarias para ello, a saber,
\begin{itemize}
\item la condición de campo débil, esto es, que el campo gravitacional en la región de interés es lo suficientemente débil.
\item la condición de movimiento lento, es decir, que el movimiento de las partículas en la zona de interés es lento en comparación con la velocidad de la luz.
\end{itemize}

La tercera condición no es tan evidente, pero se justifica a partir de hecho de que los campos gravitacionales con los que usualmente lidiamos varían muy poco con respecto al tiempo, de ahí que la tercer condición que se impondrá en la deducción es
\begin{itemize}
\item la condición de campos estacionarios, a saber, que el campo gravitacional no varía significativamente con el tiempo.
\end{itemize}

Teniendo en cuenta que la herramienta principal de trabajo será la ecuación de las geodésicas, es preciso traducir las condiciones anteriormente establecidas en términos de ecuaciones que podamos agrupar junto con la susodicha ecuación geodésica.\\
\begin{itemize}
\item La condición de campo débil se traduce en que aproximaremos la métrica del espacio-tiempo en cuentión $g_{\alpha \beta}$ por 
\begin{equation}
\label{weak field}
g_{\alpha \beta} = \eta_{\alpha \beta} + h_{\alpha \beta},
\end{equation}
donde $\eta_{\alpha \beta}$ representa la métrica de Schwarzschild y $h_{\alpha \beta}$ una pequeña perturbación, cuyo carácter pequeño se traduce en que cualquier término de orden cuadrático en $h_{\alpha \beta}$ en la deducción posterior será ignorado.
\item La condición de movimiento lento, según la cual
\begin{equation*}
\frac{dx^i}{d\tau} \ll \frac{dt}{d\tau}.
\end{equation*}
\item Finalmente, la condición de campos estacionarios, de acuerdo a la cual impondremos campos gravitacionales que no dependen del tiempo
\begin{equation*}
g_{\alpha \beta,0} =0 \hspace{.5cm}\Rightarrow\hspace{.5cm} h_{\alpha \beta,0} =0.
\end{equation*}
\end{itemize}

Considere ahora la ecuación geodésica 
\begin{equation}
\label{geodesic eqn}
\ddot{x}^\mu + \Gamma^{\mu}_{\nu \lambda}\dot{x}^\nu\dot{x}^\lambda = 0.
\end{equation}

Por la descomposición \eqref{weak field}, $\Gamma^{\mu}_{\nu \lambda}$ es por lo menos lineal en $h_{\mu \nu}$, y al imponer la condición de movimiento lento tenemos que \eqref{geodesic eqn} puede ser aproximada por
\begin{equation*}
\ddot{x}^\mu + \Gamma^{\mu}_{00}\dot{t}^2 = 0.
\end{equation*}

El carácter estacionario del campo indica que los coefcientes de Christoffel están dados por 
\begin{equation*}
\Gamma^{\mu}_{00} = -\frac{1}{2}g^{\mu \nu}\partial_\nu g_{00} = -\frac{1}{2}g^{\mu i}\partial_i g_{00}.
\end{equation*}

La condición de campo débil permite escribir
\begin{equation*}
g_{\alpha \beta} = \eta_{\alpha \beta} + h_{\alpha \beta} \hspace{.5cm} \Rightarrow \hspace{.5cm} g^{\alpha \beta} = \eta^{\alpha \beta} - h^{\alpha \beta},
\end{equation*}

donde 
\begin{equation*}
h^{\alpha \beta} = \eta^{\alpha \lambda}\eta^{\beta \rho}h_{\lambda \rho}.
\end{equation*}

Con lo anterior tenemos que 
\begin{equation*}
\Gamma^{\mu}_{00} = \eta^{\mu i}\partial_ih_{00},
\end{equation*}

de tal forma que los Christoffel de interés son
\begin{equation*}
\Gamma^{0}_{00} = 0 \hspace{.5cm},\hspace{.5cm} \Gamma^{i}_{00} = -\frac{1}{2}\eta^ih_{00}.
\end{equation*}

De esta manera, la ecuación geodésica \eqref{geodesic eqn} se divide en
\begin{equation*}
\ddot{t} = 0 \hspace{.5cm},\hspace{.5cm} \ddot{x} = \frac{1}{2}\partial_ih_{00}\dot{t}^2.
\end{equation*}

La primera ecuación indica que 
\begin{equation*}
t(\tau) = a\tau + b,
\end{equation*}

y al usar esta ecuación para convertir las derivadas con respecto a $\tau$ en derivadas con respect a $t$ tenemos que 	
\begin{equation*}
\ddot{t} = 0 \hspace{.5cm} \Rightarrow \frac{1}{\dot{t}^2}\frac{d^2}{d\tau^2} = \frac{1}{\dot{t}}\frac{d}{d\tau}\frac{1}{\dot{t}}\frac{d}{d\tau} = \frac{d^2}{dt^2}.
\end{equation*}

Por tanto se obtiene que 
\begin{equation*}
\frac{d^2x^i}{dt^2} = \frac{1}{2}h_{00,i}.
\end{equation*}

Finalmente, al comparar con la ecuación Newtoniana 
\begin{equation*}
\frac{d^2x^i}{dt^2} = -\phi_{,i}
\end{equation*}

podemos identificar 
\begin{equation*}
h_{00} = -2\phi,
\end{equation*}

entre el potencial gravitacional newtoniano y la componente tiempo-tiempo de la desviación del espacio-tiempo de la métrica de Minkowski. Al relacionar nuevamente con $g_{\alpha \beta}$ tenemos que 
\begin{equation}
\label{newton - metric}
g_{00} = - (1 + 2\phi),
\end{equation}

lo cual nos da finalmente la relación deseada entre el potencial newtoniano y la componente temporal de la métrica.\\

Teniendo en cuenta el límite newtoniano, la métrica de Hayward modificada está dada por 
\begin{equation}
\label{reg-schF}
ds^2 = -G(r)F(r) dt^2 + \frac{1}{F(r)} dr^2 + r^2d\Omega^2,
\end{equation}

donde

\begin{equation}
\label{mod-hay-f}
F(r) = 1 - \frac{Mr^2}{r^3 + 2ML^2},
\end{equation}

y $G(r)$ es un función que se explica a continuación.\\

Es evidente que solo se modifique la componente tiempo-tiempo de la métrica en tanto que la corrección efectiva de teoría cuántica de campos es única y exclusivamente al  potencial newtoniano, y dada la relación \eqref{newton - metric}, es claro que solo dicha componente deba ser modificada. Por su puesto, hay ciertas condiciones que deben imponerse sobre la función $G(r)$ de tal manera que la métrica mmodificada siga teniendo sentido físico y no se pierda el carácter regular previamente incorporado en la misma. Por tanto, las condiciones físicas que se le exigen a $G(r)$ son
\begin{itemize}
\item Preservar el comportamiento de Schwarzschild para $r \to \infty$.
\item Incluir la corrección cuántica del potencial newtoniano \eqref{newF}.
\item Permitir dilataciones de tiempo finitas entre $r \to 0$ y $r \to \infty$.
\end{itemize}

La forma más general de satisfacer las condiciones anteriormente impuestas es considerar $G(r)$ dada por

\begin{equation}
\label{mod-hay-g}
G(r) = 1 - \frac{\beta M \alpha}{\alpha r^3 + \beta M},
\end{equation}

con $\alpha$ y $\beta$ parámetros variables, sobre los cuales se imponen ciertas cotas numéricas. Un estudio más detallado de esta métrica requeriría un conocimiento más profundo acerca de la deducción de la corrección \eqref{newF}, pero adquirir ese conocimiento implicaría desviarse mucho del tema central de este trabajo, el cual es el estudio de agujeros negros regulares, por ende, se deja planteado como trabajo futuro.

\newpage
\section{Interpretación física de la métrica de Hayward}

Hasta esta instancia, solo se ha trabajado con la parte calculística inatamente asociada al estudio de agujeros negros en relatividad general y poco se ha dicho sobre la interpretación física de los modelos estudiados previamente. Dado que los cálculos teóricos son vacíos si no hay un transfondo físico detrás de los mismos, esta sección presenta una interpretación exhaustiva del modelo de interés en este documento, a saber, la métrica de Hayward.\\

Lo primero que hay que decir sobre esta métrica está relacionado con el carácter regular de la misma. Si consideramos el destino de la materia que cae dentro de un agujero negro, la relatividad general clásica indica dicha materia desaparece dentro de la singularidad espacio-temporal. No obstante, esto no es una predicción de dicha teoría, de hecho, deja entrever una falla de la misma, una incompletitud de la teoría clásica en tanto que no tiene en cuanta a la teoría cuántica. Esto da pie para pensar que correcciones cuánticas de la relatividad general clásica podrían ser un posible camino para entender de forma más feaciente lo que ocurre dentro de un agujero negro.\\

Las susodichas correcciones cuánticas que vamos a considerar en este caso provienen de cosmología cuántica \cite{rovelli}. Dos hechos fundamentales provenientes de esta teoría son:
\begin{itemize}
	\item Los efectos cuánticos en gravedad comienzan a ser dominantes cuando la densidad de energía alcanza la escala de Planck. Esta densidad está dada por 
	\begin{equation*}
	\rho_P = \frac{m_P}{l_P^3} \approx 5,155 \cdot 10^{96}\ \ kg/m^3,
	\end{equation*}
	
	donde $m_P$ y $l_P$ son la masa de Planck y la longitud de Planck, respectivamente.
	\item El efecto cuántico dominante a altas presioneses una fuerte presión, suficiente para balancear el peso de un objeto estelar y reversar su colapso.
\end{itemize}

Lo mencinado previamente sugiere qeu dicha presión gravitacional cuántica puede detener el colapso gravitacional tan pronto como la densidad de energía alcanza la escala de Planck, dando así origen a una nueva fase en la vida de un objeto que colapsa. Esta instancia es lo que se denomina estrella de Planck, y posterior a la misma el objeto que colapsa puede ``rebotar''. Una forma explícita que describe el proceso de rebote se da explícitamente en \cite{rovelliquantum}, y en este artículo se muestra que dicha métrica en efecto es una solución exacta de las ecuaciones de campo de Einstein excepto en una región cuántica compacta, limitada en espacio y tiempo, donde la teoría cuántica correige a la relatividad general clásica. No obstante, el aspecto clave es la estrella de Planck como tal, es decir, la métrica que describe la parte estacionaria de esta nueva instancia en la vida de un objeto celeste que colpsa, mas no la parte dinámica que la sigue. Esto pues toda la información realmente importante sobre el espacio-tiempo como tal está contenida en esta métrica estática y no en la parte dinámica posterior a la misma.\\

De acuerdo a \cite{ashtekar2006}, las ecuaciones de Friedman que gobiernan la dinámica del factor de escala $a(t)$ del universo es modificada por los efectos cuánticos gravitacionales de la siguiente manera
\begin{equation*}
\left(\frac{\dot{a}}{a}\right) = \frac{8\pi}{3}\rho\left(1-\frac{\rho}{\rho_P}\right)
\end{equation*}

Como se ha mencionado previamente, los efectos cuánticos en gravedad comienzan a ser dominantes cuando la densidad de energía alcanza la densidad de Planck. En relación con lo anterior, el punto es que este régimen puede ser alcanzado mucho antes de que las longitudes reñevantes en el espacio-tiempo que estemos estudiando alcancen la escala de Planck. El rebote es producido a una repulsión gravitacional-cuántica que básicamente se origina debido al principio de incertidumbre de Heissenberg, y es análoga a la ``fuerza'' que evita que el electrón colapse al interior del núcleo. Es de esperarse que este reboe ocurra cuando el universo alcanza un tamañp planckiano, pero la caracterización previamente dada solo requiere que la densidad alcance la escala de Planck y esto puede ocurrir mucho antes que lo primeramente mencionado. En un universo dominado por la materia, la densidad de Planck es alcanzada cuando el volumen del universo es
\begin{equation*}
V \approx \frac{m}{m_P}l_P^3,
\end{equation*}

donde $m$ es la masa total del universo. Por ende, el rebote puede ocurrir cuando el espacio-tiempo en cuestión es relativamente grande.\\

La relación de todo lo mencionado previamente con el estudio de agujeros negros regulares es que al entender los efectos cuánticos como correcciones de teoría cuántica de campos efectiva en relatividad general se puede resolver el problema de la singularidad física en el origen de coordenadas que presentan las métricas que describen agujeros negros. Para ver esto de forma más clara, consideremos la posibilidad de que la energía de una estrella que colapsa y cualquier energía extra que cae dentro de agujero pueda condensarse en un núcleo altamente comprimido con una densidad del orden de la densidad de Planck. Si este es el caso, el colapso gravitacional de una estrella no conlleva a una singularidad sino a la mencionada fase extra en la vida de la estrella: una fase que podríamos denominar como cuántica en la que la atracción gravitacional es balanceada por una fuerte presión gravitacional, la cual llamamos estrella de Planck. Una estrella de Planck puede tener un radio del orden de 
\begin{equation*}
r \approx \left(\frac{m}{m_P}\right)^n l_P,
\end{equation*}

donde $m$ es ahora la masa de la estrella y $n$ es un real positivo. Posteriormente se dará un estimativo para $n$, pero la pregunta impostante en este momento es si dicho modelo de las estrellas de Planck puede ser lo suficientemente estable como para dar cuenta de la larga vida de lo agujeros negros comunes. La respuesta es afirmativa pero depende radicalmente del observador que mide su tiempo de existencia, es decir, de la relatividad de los observadores en relatividad general. El periodo de vida de una estrella de Planck es muy largo si se mide desde una distancia muy grande a la estrella, esto porque dicho tiempo depende del tiempo de evaporación de Hawking del agujero negro en el cual se encuentra la estrella. Dicho tiempo es del orden de $m^3$ en unidades naturales y para un agujero negro con una masa estelar típica (muy superior a la masa del sol) este tiempo es más largo que el tiempo cosológico. Por otro lado, ese tiempo de vida es muy corto (de orden $m$, que es el tiempo que tarda la luz en atravesar la estrella) desde el punto de vista de un observador ubicado en la estrella misma.\\

Dadolo anterior, hay una gran diferencia dependiendo de la posición relativa del observador con respecto a la estrella misma, y la razón para esto es la dilatación temporal gravitacional. Es sabido que el tiempo corre más despacio en las vecidades de un fuerte campo gravitacional. En el caso hipotético de que un observador pudiera estar en la superficie de la estrella de Planck, ese se encontraría casi que inmediatamente en el futuro distante, después de que ha pasado suficiente tiempo para que el agujero haya terminado el proceso de evaporación. De esta forma es posible afirmar que el tiempo de vida propio de una estrella de Planck es corto desde su propia perspectiva, y la estrella es esencialmente un rebote. Por ende, un agujero negro sería, de acuerdo a estos razonamientos, un atajo hacia el futuro distante.\\

Ahora bien, si esto es lo que realmente pasa en la naturaleza, el interior de un agujero negro formado por una estrella que ha colapsado gravitacionalmente puede ser modelado por euna métrica efectiva que es solución exacta de las ecuaciones de campo de Einstein en el exterior de la región planckiana, pero no en el interior de la misma, debido a los efectos cúánticos previamente aludidos. Las propiedades de una métrica así se pueden visualizar de la siguiente manera. En las coordenadas de Eddington-Finkelstein, los conos de luz, los cuales se doblan hacia adentro en el horizonte externo del agujero, vuelven a apuntar hacia arriba de forma continua cuando entran en la región planckiana. De esta manera, hay un segundo horizonte atrapado dentro del horizonte de Schwarzschild, a una escala $r_{in}$ relacionda con la escaloa de Planck. Esto puede verse gráficamente en la fig. \ref{fig: light cones}.
\begin{figure}[h!]
	\centering
	\includegraphics[width=0.35\textwidth]{lightcones}
	\caption{Ilustración del doblamiento hacia adentro y hacia arriba en los dos radios mencionados, para una estrella de Planck en coordenadas de Eddigton-Finkelstein. Imagen tomada de \cite{rovelli}.}
	\label{fig: light cones}
\end{figure}

Daremos ahora una aproximación del parámetro $n$ mencionado unas líneas más ariiba. En términos dimensionales, la curvatura (escalar) se comporta como
\begin{equation}
\label{curvature1}
R \approx \frac{m}{r^3},
\end{equation}

donde $r$ es la corrdenada radial de Schwarzchild. Aquí $m$ es la masa de la estrella colapsada, que coincidecon la masa total del agujero negro medida desde el infinito. A partir de las ecuaciones de Einstein, la curvatura es proporcional a la densidad de energía y la hipótesis realizada por Rovelli y Vidotto en \cite{rovelli} es que cuando la curvatura alcanza la escala de Planck comienza el régimen de los efectos cuánticos graviacionesles, y matemáticamente esta hipótesis se traduce en que
\begin{equation}
\label{curvature2}
R \approx 8\pi\rho_P.
\end{equation}

A partir de las ecuaciones \eqref{curvature1},\eqref{curvature2} tenemos que el borde de la región no-clásica está dado por
\begin{equation*}
r_{in} \approx \left(\frac{m}{m_P}\right)^{\frac{1}{3}}l_P.
\end{equation*}

Esto nos permite dar el aproximado $n \sim 1/3$. Un ejemplo particular de una méttrica que describe, precisamente, las estrellas de Planck al interior de una agujero negro es la métrica de Hayward, la cual mimetiza la fuerza gravitacional-cuántica repulsiva que en última permite que se evite la singularidad, en este caso, en la métrica RN. Esto se puede ver con claridad al recordar que la expansión de $f(r)$ cuando $r \to \infty$ es
\begin{equation*}
f(r) \underset{r \to \infty}{\sim} 1 - \frac{2m}{r} + \frac{4l^2m^2}{r^4} + \mathcal{O}(r^{-5}).
\end{equation*}

Los dos primeros términos son los usuales de cualquier métrica asintóticamente plana, por ende, corresponden a los términos usuales del potencial gravitacional newtoniano (ley de la gravitación universal de Newton); el último término mostrado representa una fuerte fuerza repulsiva de corto alcance debida a las correcciones de teoría cuántica de campos efectiva en relatividad general.\\

Ayende de lo anterior, es preciso explicar el carácter regular de la métrica de Hayward a la luz de l teorema de Borde. Como esta métrica está inspirada en la métrica RN, su diagrama de Carter-Penrose es muy similar, pero dado el carácter regular de $r = 0$, dicho punto en el diagrama deja de ser singular y pasa a ser un punto oridinario de este espacio-tiempo. En la fig. \ref{fig: carter penrose diagram hayward} se puede apreciar una porción de este diagrama.\\
\begin{figure}[h!]
	\centering
	\includegraphics[width=0.25\textwidth,height=0.34\textheight]{CPDhayward}
	\caption{Diagrama de Carter-Penrose para la métrica de Hayward estática. Imagen tomada de \url{https://www.physicsforums.com/threads/penrose-diagram-for-the-kerr-black-hole.789730/}.}
	\label{fig: carter penrose diagram hayward}
\end{figure}

Lo primero que se debe hacer s verificar que se satisfacen las cuatro hipótesis necesarias para que se satisfaga el teorema de Borde. La primera hipótesis es que el espacio-tiempo contuviera una superficie eventualmente futuramente atrapada (véase la def. \ref*{trapped surfaces}). Como se mencionó previamente, esta condición es satisfecha casi que automáticamente en cualquier espacio-tiempo y en este caso la superficie eventualmente futuramente atrapada corresponde con el horizonte externo de la métrica de Hayward. La segunda hipótesis es la condición de convergencia nula, la cual también se satisface en este caso puesto que ya se vió que el espacio-tiempo descrito por la métrica de Hayward satisface la condición de energía débil. La tercera hipótesis es que sea geodésicamente-nulo completo hacia el futuro, que básicamente significa que el espacio-tiempo no contenga singularidades, y como se vió previamente la métrica de Hayward es regular en todos los puntos del espacio-tiempo. Solo resta verificar la última hipótesis, pero dada la importancia de la misma, se tratará con un mayor cuidado.\\

La última hipótesis del teorema de Borde establece que el universo sea causalmente simple, y como la métrica de Hayward está inspirada en la métrica de Bardeen, esta condición se satisface dado que la estructura causal del espacio-tiempo que general es semejante a la de Schwarszchild.\\

Más allá de esto, el carácter regular de la métrica de Hayward indica que cualquier geodésica proveniente del interior del horizonte de este agujero negro regular debe alcanzar la región $r \to \infty$. Esto puede realizarse de multiples formas, entre ellas destacan

\begin{itemize}
\item Identificando las dos ramificaciones correspondientes a $r = 0$.

\item Identificando la ramificiación $r = 0$ de la derecha con $r = r_+$ en la parte superior derecha de la figura \ref{fig: carter penrose diagram hayward}.

\item Identificando la ramificiación $r = 0$ de la derecha con $r = r_+$ en la parte superior izquierda de la figura \ref{fig: carter penrose diagram hayward}.
\end{itemize}

Con las primeras dos identificaciones, el espacio-tiempo en cuestión sigue siendo orientable, mientas que con la tercera identificación, esto no es posible dado a que las geodésicas  futuramente orientadas cambian su dirección. Además, con la segunda identificación estaríamos excluyendo toda una parte del espacio-tiempo, que es la que contiene el punto que anteriormente era una singularidad, y no hay ninguna justificación física para excluir esta región. Por ende, es conveniente trabajar con la primera identificación.
\newpage
\section{\label{conclusions} Conclusiones}

\newpage
%\nocite{*}
\bibliographystyle{unsrt}
\bibliography{Biblio}


\end{document} 