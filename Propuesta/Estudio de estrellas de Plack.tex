\documentclass[12pt]{article}

\usepackage{graphicx}
\usepackage{epstopdf}
\usepackage[spanish]{babel}
%\usepackage[english]{babel}
%\usepackage[latin5]{inputenc}
\usepackage{hyperref}
\usepackage[left=1.9cm,top=3cm,right=3cm,nohead,nofoot]{geometry}
\usepackage{braket}
\usepackage{datenumber}
%\newdate{date}{10}{05}{2013}
%\date{\displaydate{date}}

\begin{document}

\begin{center}
\Huge
Estudio de estrellas de Planck 

\vspace{3mm}
\Large Alejandro Hern\'andez A.

\large
201219580


\vspace{2mm}
\Large
Director: Pedro Bargue\~no de Retes

\normalsize
\vspace{2mm}

\date{}
\end{center}


\normalsize
\section{Introducci\'on}

%Introducci�n a la propuesta de Monograf�a. Debe incluir un breve resumen del estado del arte del problema a tratar. Tambi�n deben aparecer citadas todas las referencias de la bibliograf�a (a menos de que se citen m�s adelante, en los objetivos o metodolog�a, por ejemplo)

El conocimiento actual del funcionamiento de la gravedad se basa en la Teor\'ia General de la Relatividad de Einstein, cuya formulaci\'on matem\'atica m\'as general son las ecuaciones de campo, a saber:

\begin{equation}
\label{campo}
R_{\mu \nu} - \frac{1}{2} R g_{\mu \nu} + \Lambda g_{\mu \nu} = 8 \pi T_{\mu \nu}
\end{equation}

donde $R_{\mu \nu}$ es el tensor de Ricci, $R$ es la curvatura escalar, $T_{\mu \nu}$ es el tensor energ\'ia-momento y $\Lambda$ es la constante cosmol\'ogica.\\

Muchas de las soluciones de \ref{campo} no est\'an bien definidas en todo el espacio-tiempo, tal y como ocurre, por ejemplo, con la m\'etrica de Schwarzschild \ref{sch}  en $r = 0$.

\begin{equation}
\label{sch}
ds^2 = -\left( 1 - \frac{2M}{r} \right) dt^2 + \left( 1 - \frac{2M}{r} \right)^{-1} dr^2 + r^2d\Omega ^2
\end{equation}

Una posible forma de solucionar estos problemas podr\'ia ser considerar correcciones cu\'anticas a la teor\'ia gravitacional. Adem\'as, el hecho de que todos los campos no gravitacionales est\'en bien descritos por la teor\'ia cu\'antica de campos, hace pensar que existe una teor\'ia cu\'antica subyacente para la gravedad.\\ 

Una forma de implementar de forma efectiva correcciones cu\'anticas en la m\'etrica de Schwarzschild es mediante la denominada m\'etrica de Hayward modificada \cite{hayward,effective}, que viene dada por \ref{reg-sch} 

\begin{equation}
\label{reg-sch}
ds^2 = -G(r)F(r) dt^2 + \frac{1}{F(r)} dr^2 + r^2d\Omega ^2
\end{equation}

donde

\begin{equation}
\label{reg-f}
F(r) = 1 - \frac{Mr^2}{r^3 + 2ML^2}
\end{equation}

\begin{equation}
\label{reg-g}
G(r) = 1 - \frac{\beta M \alpha}{\alpha r^3 + \beta M}
\end{equation}

con $\alpha$ y $\beta$ par\'ametros del sistema. Esta m\'etrica (basada en los argumentos dados en \cite{hayward,bardeen}) representa lo que usualmente es referido en la literatura como \emph{Estrellas de Planck} \cite{planck stars} y  satisface las siguientes propiedades:

\begin{itemize}
\item $g_{00} = 1 - \frac{2M}{r}$ para $r \rightarrow \infty$, es decir, la m\'etrica es asint\'oticamente Schwarzschild.

\item Incluye correcciones efectivas de teor\'ia cu\'antica de campos al potencial Newtoniano dadas por 

\begin{equation}
\label{new}
\Phi (r) = -\frac{M}{r} \left( 1 + \beta \frac{l_{p}^2}{r^2} \right) + \mathcal{O}(r^{-4}),
\end{equation}

donde $l_{p}$ en la longitud de Planck.
\item Permite una dilataci\'on temporal finita entre $r = 0$ y $r = \infty$.

\item $g_{00} = 1 - \frac{r^2}{L^2} + o(r^3)$, es decir, es de Sitter para $r \rightarrow 0$.

\end{itemize}


Las motivaciones f\'isicas para proponer la m\'etrica \ref{reg-sch}, adem\'as de las consecuencias y propiedades de la misma son de gran inter\'es te\'orico y ejemplifican una forma particular de incluir efectos cu\'anticos de manera efectiva para regularizar la soluci\'on de Schwarzschild.  
\section{Objetivo General}

%Objetivo general del trabajo. Empieza con un verbo en infinitivo.

Estudiar en detalle la estructura de las estrellas de Planck y comprender las correcciones cu\'anticas implementadas en la m\'etrica \ref{reg-sch}.


\section{Objetivos Espec\'ificos}

%Objetivos espec�ficos del trabajo. Empiezan con un verbo en infinitivo.

\begin{itemize}
	\item Entender (y calcular, en la medida de lo posible) las correcciones cu\'anticas del potencial Newtoniano $\Phi (r)$ mostradas en \ref{new}.
	\item Estudiar las m\'etricas de Bardeen y de Vaidya como punto de partida para la regularizaci\'on de agujeros negros \cite{vaidya}.
	\item Comprender la importancia de la regularizaci\'on de la m\'etrica de Schwarzschild.
	\item Entender las consecuencias de la regularizaci\'on mencionada previamente y el mecanismo que impide la formaci\'on de la singularidad.
	\item Entender la introducci\'on de una dilataci\'on temporal finita entre $r = 0$ y $r = \infty$.
	\item Tratar de generalizar el procedimiento de regularizaci\'on a la m\'etrica de Reissner-Nordstr\"om.
\end{itemize}

\section{Metodolog\'ia}

%Exponer DETALLADAMENTE la metodolog�a que se usar� en la Monograf�a. 

%Monograf�a te�rica o computacional: �C�mo se har�n los c�lculos te�ricos? �C�mo se har�n las simulaciones? �Qu� requerimientos computacionales se necesitan? �Qu� espacios f�sicos o virtuales se van a utilizar?

%Monograf�a experimental: Recordar que para ser aprobada, los aparatos e insumos experimentales que se usar�n en la Monograf�a deben estar previamente disponibles en la Universidad, o garantizar su disponibilidad para el tiempo en el que se realizar� la misma. �Qu� montajes experimentales se van a usar y que material se requiere? �En qu� espacio f�sico se llevar�n a cabo los experimentos? Si se usan aparatos externos, �qu� permisos se necesitan? Si hay que realizar pagos a terceros, �c�mo se financiar� esto?

La metodolog\'ia usada ser\'a un trabajo aut\'onomo del estudiante, con seguimiento continuo por parte del director mediante reuniones semanales en las que se revise el cumplimiento de las tareas establecidas en el cronograma mostrado posteriormente, se resuelvan dudas del estudiante, se proponga nueva bibliograf\'ia en caso de ser necesario, y se establezcan objetivos particulares para la siguiente reuni\'on.\\

El trabajo mencionado en el p\'arrafo anterior consiste en revisiar la bibliograf\'ia propuesta, realizar los c\'alculos te\'oricos necesarios (especificados en la secci\'on \textbf{Cronograma}) y redactar el documento final de la monograf\'ia. Cuando sea necesario, se usar\'a el software \emph{Mathematica}, dependiendo de la extensi\'on de los c\'alculos.

\section{\label{crono}Cronograma}

\begin{table}[htb]
	\begin{tabular}{|c|cccccccccccccccc| }
	\hline
	Tareas $\backslash$ Semanas & 1 & 2 & 3 & 4 & 5 & 6 & 7 & 8 & 9 & 10 & 11 & 12 & 13 & 14 & 15 & 16  \\
	\hline
	1 & X & X &   &   &   &   &   &  &  &   &   &   &   &   &   &   \\
	2 &   & X & X &   &  &  &  &   &   &  &  &  &   &  &  &   \\
	3 &   &   & X & X &  X & X  & X &  &   &   &   &  &   &   &  &   \\
	4 &   &   &   &   & X & X & X & X & X & X &   &   &   &   &   &   \\
	5 &   &   &   &   &  &   & X  & X  & X & X  & X  & X &   &   &  &   \\
	6 &   &   &   &   &  &   &   & X  & X &   &   &  &   &   &  &   \\
	7 &   &   &   &   &  &   &   &   & X & X  & X  & X &   &   &  &   \\
	8 &   &   & X  & X  & X & X  & X  &  X & X &  X & X  & X & X  & X  & X & X  \\
	\hline
	\end{tabular}
\end{table}
\vspace{1mm}

\begin{itemize}
	\item Tarea 1: Recordar la formulaci\'on tensorial de la Relatividad General y algunos teoremas de singularidad \cite{inverno}.
	\item Tarea 2: Recordar la definici\'on de horizontes de eventos y horizontes de Killing \cite{hawking}.
	\item Tarea 3: Estudiar las m\'etricas de de Sitter, Bardeen y Vaidya.
	\item Tarea 4: Estudiar y comprender las correcciones cu\'anticas al potencial gravitacional Newtoniano.
	\item Tarea 5: Estudiar la m\'etrica de Hayward modificada y calcular el tensor energ\'ia momento asociada a ella .
	\item Tarea 6: Redactar y presentar el documento de avance de la monograf\'ia. 
	\item Tarea 7: Calcular y entender la dilataci\'on temporal entre $r = 0$ y $r = \infty$, adem\'as del invariante de Kretschmann para \ref{reg-sch} y verificar que se evita la singularidad en $r = 0$.
	\item Tarea 8: Redactar el documento final de la monograf\'ia.
\end{itemize}

\section{Personas Conocedoras del Tema}

%Nombres de por lo menos 3 profesores que conozcan del tema. Uno de ellos debe ser profesor de planta de la Universidad de los Andes.

\begin{itemize}
	\item Andr\'es Fernando Reyes Lega (Universidad de los Andes, Departamento de F\'isica)
	\item Alonso Botero Mej\'ia (Universidad de los Andes, Departamento de F\'isica)
	\item Marek Nowakowski (Universidad de los Andes, Departamento de F\'isica)
\end{itemize}


\begin{thebibliography}{10}

\bibitem{hayward} Hayward, S.A.: Formation and Evaporation of regular black holes. Phys. Rev. Lett. \textbf{96}, 31103 (2006).

\bibitem{effective} De Lorenzo, T., Pacilio, C., Rovelli, C., Speziale, S.: On the effective metric of a Planck star. Gen. Relativ. Gravit. \textbf{47}, 41 (2015).

\bibitem{bardeen} Bardeen, J.M.: Non-singular general-relativistic gravitational collapse. In: Procceedings of International Conference GR5, Tbilisi, p. 174 (1968).

\bibitem{planck stars} Rovelli, C., Vidotto, F.: Planck Stars. Int. J. Mod. Phys. D. \textbf{23}, 1142026, (2014).


\bibitem{vaidya} Padmanabhan, T.: \textit{Gravitation: Foundations and Frontiers}. Cambridge University Press, Cambridge (2010).

\bibitem{inverno} d'Inverno, R. A.: \textit{Introducing Einstein's Relativity}. Oxford University Press, New York (1992).

\bibitem{hawking} Hawking, S.W., Ellis, G.F.R.: The Large Scale Structure of Space-time, vol. 1, 20th edn. Cambridge University Press, Cambridge (1973).


\end{thebibliography}

\section*{Firma del Director}
\vspace{2.5cm}

\section*{Firma del Estudiante}



\end{document} 